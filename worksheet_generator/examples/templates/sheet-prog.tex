\documentclass[10pt]{scrartcl}
\usepackage[a4paper,total={15.24cm, 20.32cm}, top=1.5cm, left=1.5cm, right=1.5cm, bottom=2cm]{geometry}
\usepackage{amsmath,amsthm,amssymb,amscd,latexsym}
\usepackage{amsfonts}
\usepackage{mathrsfs}
\usepackage[T1]{fontenc}
\usepackage[utf8]{inputenc} %unter Linux
\usepackage{inconsolata} %gives slashed zeros
\usepackage{graphicx}
\usepackage{import}
\usepackage{verbatim}
\usepackage[linesnumbered]{algorithm2e}

\usepackage{tikz}
\usetikzlibrary{arrows,automata}

\usepackage{chancery}  % font family
\renewcommand{\familydefault}{\sfdefault} 

\usepackage{hyperref} % colors of links
\hypersetup{
	colorlinks=true,%, %set true if you want colored links
	linkcolor=blue,  %choose some color if you want links to stand out
	citecolor=blue,
	urlcolor=blue
}

\usepackage{xcolor} 
\usepackage{color} % for \definecolor
\definecolor{gridgray}{rgb}{0.9, 0.9, 0.9}
\definecolor{codegreen}{rgb}{0,0.6,0}
\definecolor{codegray}{rgb}{0.5,0.5,0.5}
\definecolor{codepurple}{rgb}{0.58,0,0.82}
\definecolor{backcolour}{rgb}{0.95,0.95,0.92}
\definecolor{python}{rgb}{0.0,0.0,0.0}
\definecolor{darkgray}{rgb}{0.3,0.3,0.3}
\definecolor{gray}{rgb}{0.6,0.6,0.6}
\definecolor{navy}{rgb}{0.0,0.0,0.5}
\definecolor{solution}{rgb}{0.0,0.4,0.0}
\definecolor{exampoints}{rgb}{1.0,0.0,0.0}

\usepackage{listings}
\lstdefinestyle{python}{
	language=python,
	backgroundcolor=\color{backcolour},   
	commentstyle=\color{codegreen},
	keywordstyle=\color{blue},
	numberstyle=\tiny\color{codegray},
	stringstyle=\color{codepurple},
	basicstyle=\small\ttfamily,%\footnotesize,
	breakatwhitespace=false,         
	breaklines=true,                 
	captionpos=b,                    
	keepspaces=true,                 
	numbers=left,                    
	numbersep=5pt,                  
	showspaces=false,                
	showstringspaces=false,
	showtabs=false,                  
	tabsize=2,
		    extendedchars=true,
	literate={ä}{{\"a}}1 {ü}{{\"u}}1 {ö}{{\"o}}1 {ß}{{ss}}1 {Ü}{{\"U}}1
}
\lstset{style=python}

% PAGE SETUP
\pagestyle{empty}
\setlength{\parindent}{0pt}
\setcounter{page}{0}



%%%%%%%%%%%%%%%%%%%%%%%%%%%%%%%%%
\usepackage{exercisecommand}
%%%%%%%%%%%%%%%%%%%%%%%%%%%%%%%


% VARIABLES (GENERATED BY PYTHON)
\newcommand{\LectureName}{Elements of Mathematics}
\newcommand{\Lecturer}{Dr. Christian Vollmann}
\newcommand{\Semester}{Winter Term 2022}
\newcommand{\SheetTitle}{Sheet 01}
\newcommand{\Date}{Due date: \textbf{XXX}}
\newcommand{\Tutor}{}
\newcommand{\Place}{}


\begin{document}
	
%%%%%%%%%%%%% HEADER %%%%%%%%%%%%%%%%%%
\textit{
	\hspace*{-0.1cm}\Lecturer\hfill \Semester\\\Tutor\textcolor{white}{a}\hfill\text{\color{black}\tiny [\today]}\\
}
\begin{large}
	\begin{center}
		\textbf{\Large\LectureName}\\ 
		\SheetTitle
		~\\~\\
		\normalsize \text{\color{red}\Date}\\
		\Place
	\end{center}
\end{large}
\vspace{1cm}

%%%%%%%%%%%%% ID FIELD %%%%%%%%%%%%%%%%%%
\begin{center}
	\begin{minipage}{0.7\textwidth}
	\hrule 
	\vspace*{0.3cm}~\\
	{ \large
		%		\begin{center}
		Name: \vspace*{0.6cm}~\\
		Matriculation number:  \\
		%		\end{center}
	}
	\hrule
	\vspace{1cm}
\end{minipage}
\end{center}

%%%%%%%%%%%%% TEXT %%%%%%%%%%%%%%%%%%
\begin{center}
	\begin{minipage}{0.8\textwidth}
		\textit{“You should name a variable using the same care with which you name a first-born child.”} 
		
		{\color{darkgray}
			― Robert C. Martin, Clean Code: A Handbook of Agile Software Craftsmanship 
		}
		\vspace{.5cm} \\
		
		Allgemeine Hinweise:
		\begin{itemize}
			\item Mit [n] gekennzeichnete Aufgaben können noch in n verschiedenen Übungen vorgerechnet werden. Mehrmalige Präsentation einer Aufgabe in derselben Übungsstunde ist möglich.
			\item Wenn Sie Schwierigkeiten mit der Installation von Anwendungsprogrammen haben, kommen Sie damit gerne zur nächsten Übungsstunde. Gerne können wir in dieser Stunde auch Lösungen für Aufgaben gemeinsam erarbeiten.
		\end{itemize}
		Hinweise zur Programmierung
		\begin{itemize}
			\item Verwenden Sie \textbf{sprechende Namen für die genutzten Variablen und Funktionen}. Nehmen Sie sich Zeit für die Namensfindung.
			\item Schauen Sie sich den  \href{https://peps.python.org/pep-0008/#function-and-variable-names}{\textbf{Style Guide für Python Code}} an und setzen Sie die Regeln bezüglich der \textbf{Bennenung von Variablen und Funktionen} um.
			\item \textbf{Testen} Sie stets Ihre Programme und Programmteile. Tests können beispielsweise bekannte bzw. erwartete (Input, Output)--Paare für eine Funktion sein.
		\end{itemize}
	\end{minipage}
\end{center}

%%%%%%%%%%%%%%%%%%%%%%%%%%%%%%%%%%%%
% GENERATED BY PYTHON:
%%%%%%%%%%%%%%%%%%%%%%%%%%%%%%%%%%%%

%%%%%%%%%%%%% POINTS TABLE %%%%%%%%%%%%%%%%%%
\vspace*{1cm}~\\
\begin{center}
	\Large
	\begin{longtable}[]{@{}lll@{}}
\toprule
time & precision & algorithm\tabularnewline
\midrule
\endhead
0.1 & 0.01 & A\tabularnewline
1.0 & 0.00234 & B\tabularnewline
10.0 & 8.98e-05 & C\tabularnewline
\bottomrule
\end{longtable}

\end{center}

%%%%%%%%%%%%% EXERCISES %%%%%%%%%%%%%%%%%%
\newpage
\setcounter{page}{1}
\exercises{../../../../database/programming/FloatsHalfPrecision/}{../../../../database/programming/FloatsHalfPrecision/}{}{5}{true}{true}{false}
\exercises{../../../../database/programming/Permutation/}{../../../../database/programming/Permutation/build/}{}{5}{true}{true}{false}
\exercises{../../../../database/programming/Maximum/}{../../../../database/programming/Maximum/build/}{}{7}{true}{true}{false}
\exercises{../../../../database/programming/CallbackFunction/}{../../../../database/programming/CallbackFunction/build/}{}{5}{true}{true}{false}
\exercises{../../../../database/programming/DezToAnyConversion/}{../../../../database/programming/DezToAnyConversion/build/}{}{8}{true}{true}{false}
\exercises{../../../../database/programming/EuklidischerAlgorithmus/}{../../../../database/programming/EuklidischerAlgorithmus/build/}{}{5}{true}{true}{false}
\exercises{../../../../database/programming/Fakultaet/}{../../../../database/programming/Fakultaet/build/}{}{2}{true}{true}{false}
\exercises{../../../../database/programming/MaxSort/}{../../../../database/programming/MaxSort/build/}{}{5}{true}{true}{false}
\exercises{../../../../database/programming/Heron/}{../../../../database/programming/Heron/build/}{}{5}{true}{true}{false}

\end{document}

