\textbf{Gram-Schmidt-Algorithm}\\
{\color{navy}
	The Gram-Schmidt algorithm is an algorithm to compute a QR-decomposition of some matrix $A$.
	The basic idea is to successively 
	built up an orthogonal system from a given set of linearly independent vectors. Those are, in this case, given by the columns of an invertible matrix 
	$A = [a_1, \dots, a_n] \in \mathbb{R}^{n\times n}$.
	We choose the first column as starting point for the algorithm and set $\widetilde{q_1} := a_1$. 
	Of course, in order to generate an orthogonal matrix $Q$ we have to rescale the vector and set
	$q_1 := \frac{\widetilde{q_1}}{\| \widetilde{q_1} \|}$.
	The successive vectors $\widetilde{q}_k$ are generated by
	subtracting all the shares $a_k^\top q_\ell$ of the previous vectors $q_\ell$ from the column $a_k$, i.e.
	$$
	\widetilde{q}_k := a_k - \sum_{\ell = 1}^{k-1} a_k^\top q_\ell \,\, q_\ell.
	$$
}
The following algorithm computes a QR-decomposition of some matrix $A \in \mathbb{R}^{n \times n}$.
\begin{algorithm}
	$r_{11} \gets \| a_1 \|$\;
	$q_1 \gets \frac{a_1}{r_{11}}$\;
	\For{$k = 2, \dots, n$}{ 
		\For{$\ell = 1,\dots, k-1$}{
			$r_{\ell k} \gets a_k^\top q_\ell$	
		}
		$\widetilde{q}_k \gets a_k - \sum _{\ell=1}^{k-1} r_{\ell k} \, q_\ell$\;
		$r_{kk} \gets \| \widetilde{q}_k \|$\;
		$q_k \gets \frac{\widetilde{q}_k}{r_{kk}}$\;
	}
	\caption{Gram-Schmidt algorithm}
\end{algorithm}
\begin{enumerate}
	\item Please check that the matrix $Q := [q_1, \dots, q_n] \in \mathbb{R}^{n \times n}$ is orthogonal, i.e. that $Q^TQ = I_n$.
	
	\textit{Hint: } Show $\| q_i\| = 1$ first, and then perform an induction proof using the induction assumption that $q_k$ is orthogonal to
	all previous $ q_1,\dots, q_{k-1}$.
	
	\item Let $R := \left(r_{\ell k}\right)_{\ell \leq k}$ be the upper triangular matrix which results from the Gram-Schmidt algorithm. Please show that the algorithm provides a QR decomposition, i.e., that $QR = A$.
	
	\textit{Hint: } It suffices to show $Q r_k = a_k$, where $r_k$ ($a_k$) is the $k$-th column of $R$ (A).
\end{enumerate}