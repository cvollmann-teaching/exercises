\textbf{Maximum einer Liste}\\
Die Funktion \verb|maximum(values)| gibt das maximale Element der Liste \verb|values| zurück.
\begin{enumerate}
	\item Implementieren Sie \verb|test()| und \verb|maximum(values)|. 
	\item Die Funktion \verb|maximum| soll auch prüfen, ob die eingegebene Liste mindestens ein Element enthält und anderenfalls einen Fehler produzieren.
	Probieren sie aus, was der Python Befehl
	\begin{lstlisting}
		assert Wert "Hier gibt es ein Problem!"\end{lstlisting}
	für \verb|Wert=False| oder \verb|True| tut und versuchen Sie ihn zu nutzen.
	\item \textbf{Bonus: } Was bewirkt die (optimierte) Ausführung des Programms via \verb|python -O| hier?
\end{enumerate}
\textbf{Hintergrund: } Im Bereich des wissenschaftlichen Rechnens sind Fehlertoleranz und Performanz manchmal Gegensätze. Die elganteste Synthese dieser Gegenspieler ist dann ein Code, der sich umschalten lässt. \\
