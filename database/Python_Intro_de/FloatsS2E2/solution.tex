% !TeX spellcheck = de_AT_frami
(Vergleiche Kapitel 4.4.1)
	\begin{enumerate}
		\item  M = Länge Exponentenbitfolge = 2\\
		 also  $$B= 2^{M-1}-1 = 2^{2-1}-1 = 1$$
		mögliche Bitmuster für Exponent:
		\begin{itemize}
			\item 00 reserviert für subnormals  
			\item reserviert für inf, NaN
			\item 10 (=2), 01 (=1) 
		\end{itemize} 
		Also 
		\begin{itemize}
			\item  $e_{\text{min}} = (01)_2 - B = 1- 1 = 0$
			\item $e_{\text{max}}= (10)_2 - B = 2- 1 = 1$ 
		\end{itemize}
		\item  5 Bits: 1 Vorzeichen, 2 Exponent, 2 Mantisse \\ in der Reihenfolge: \texttt{VEEMM}%, wobei \texttt{VEEMM}=$(-1)^V\cdot (1.MM)_2 \cdot 2^{(EE)_2 - B}$
		\item N = Mantissenlänge = 2, also $$\texttt{macheps}=(\frac{1}{2})^2=\frac{1}{4}.$$
		\item Geben Sie die Bitmuster mit zugehörigem Zahlenwert an:
		\begin{enumerate}
			\item signed zero (EE=00): +0 = [00000], -0 = [10000]\\
			signed infinity (EE=11):  +$\infty$=[01100], -$\infty$=[11100]
			\item Kleinste positive normale Zahl = [00100] = 1.0\\
			Größte positive normale Zahl = [01011] = 3.5
			\item Kleinste  positive denormalisierte Zahl = [00000] = 0 \\
			Größte positive denormalisierte Zahl = [00011] = 0.75
		\end{enumerate}
	\end{enumerate}

