\textbf{Rot13: Eine Caesar-Verschlüsselung}\\
Das \href{https://de.wikipedia.org/wiki/ROT13}{Rot13} Verfahren ist eine Vertauschungsmethode für Buchstaben. Dabei wird jeder Buchstabe um 13 stellen weiter gerückt (z.B. \verb|A --> N|).

\begin{enumerate}
	\item Was passiert, wenn ein \textit{vertauschter} Text erneut in die Funktion eingegeben wird? Welcher Eigenschaft einer Permutation entspricht das?
	\item Schreiben Sie einen Test \verb|test_rot13()|.
	\item Schreiben sie eine Python-Funktion \verb|rot13(message)|, die einen string \texttt{message} entgegennimmt und darin jeden Buchstaben um 13 Stellen verschiebt.
	Gehen Sie hierbei gerne davon aus, dass die Eingabe nur aus Großbuchstaben (ohne Umlaute) besteht. 
\end{enumerate}

\textit{Hinweis: } Dictionaries erlauben hier eine sehr einfach Implementierung.