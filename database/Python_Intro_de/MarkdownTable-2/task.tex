 \textbf{Markdown--Tabellen}\\
Mit Markdown können wir leicht \href{https://www.tablesgenerator.com/markdown_tables}{Tabellen} zeichnen. 
\begin{enumerate}
\item Schreiben Sie selbst eine Funktion \verb|tabulate(valuedict)|, dass ein Dictionary entgegennimmt und einen String zurück gibt, der als Markdown--Tabelle interpretiert werden kann.
\item Testen Sie automatisiert \textit{und} mit einem Markdown Interpreter, wie er beispielsweise in den Jupyter Notebooks zu finden ist.
\item Schauen Sie sich das Python--Paket \verb|tabulate| an und stellen Sie es kurz vor.
%
%\item \textbf{Bonus}: Speichern Sie den String \verb|tabulate(valuedict)| zusätzlich in eine Textdatei \texttt{filename.md}.\\ 
%\textit{Hinweis:} Sie benötigen dazu ein \href{https://docs.python.org/3/tutorial/inputoutput.html#reading-and-writing-files}{Dateiobjekt}, welches Sie leicht mit der eingebauten Funktion \texttt{open()} erstellen. Dann können Sie die Methode \texttt{.write()} zum schreiben verwenden und die Methode \texttt{.close()}, um die Datei zu speichern und zu schließen.
\end{enumerate}