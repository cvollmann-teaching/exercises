\begin{enumerate}
	\item s10e5 bedeutet M=5, N=10. Damit erhalten wir \\
	$B = 2^5 - 1= 15$,\\
	$e_{min} = (00001)_2 - B = 1-15 = 14$,\\
	$e_{max} = (11110)_2 - B = 30-15 = 15$
	\item Vorzeichenbit, Exponent (5 Bits), Mantisse (10 Bits) insgesamt 16 Bit in der Reihenfolge
	$$\texttt{VEEEEEMMMMMMMMMM}.$$
	 Der Zahlenwert ergibt sich abhängig vom Exponentenbitmuster; siehe Abschnitt 4.4.1, insbesondere Folge 97.
	 \item N = Mantissenlänge = 10, also $$\texttt{macheps}=\left(\frac{1}{2}\right)^{10}  = 0.0009765625.$$
	\begin{enumerate}
			\item siehe  \href{https://en.wikipedia.org/wiki/Half-precision_floating-point_format#Half_precision_examples}{Wikipedia}
			\item Wikipedia
			\item Wikipedia
		\item NaN (EEEEE=11111)\\
			kleinste positive NaN =
			\texttt{0111110000000001}\\
			größte positive NaN =
\texttt{0111111111111111}
			\item z := \texttt{0011110000000001}\\
			V = 0, also positiv\\
			EEEEE=01111, also normal und realer Exponent $e = (01111)_2 - B = 15-15 = 0 $\\
			Mantisse = $m=(1.0000000001)=1+\left(\frac{1}{2}\right)^{10} = 0.0009765625 \approx 1.001$\\
			also $z = (-1)^V \cdot m \cdot 2^e \approx 1.001$
	\item $\texttt{fl}(-4)$\\
	$4 = (100)_2 = (1.{\color{orange}0000000000})_2 \cdot 2^2$\\
	Also realer Exponent $e = 2 = E - B = E - 15 ~\iff~$ verschobener Exponent $E=17=(10001)_2$\\
	Mit Vorzeichen V=1, erhalten wir
	$$fl(-4)= -4 = \texttt{110001{\color{orange}0000000000}} $$
	\item $\texttt{fl}(\frac{2}{3})= \texttt{fl}((0.10101010\ldots)_2)=(1.{\color{orange}0101010101})_2\cdot 2^{-1}$\\
	positiv also $V=0$\\
	realer Exponent $e = -1 = E - B = E - 15 ~\iff~$ verschobener Exponent $E=14=(01110)_2$\\
	insgesamt also 
		$$\texttt{fl}(\frac{2}{3})=\texttt{001110{\color{orange}0101010101}} $$
	\end{enumerate}
\end{enumerate}
