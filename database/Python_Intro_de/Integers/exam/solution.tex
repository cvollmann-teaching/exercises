% !TeX spellcheck = de_AT_frami
\begin{enumerate}
	\item Berechnen Sie die folgenden Summen im entsprechenden Stellenwertsystem.
	\begin{enumerate}
		\item \textcolor{red}{(1P)} $(10101010)_2$ (170) + $(11100000)_2$ (224) = $(110001010)_2$  (394)\\
		hier: $C=\texttt{carry(N-1,N)}=1,~~~ c=\texttt{carry(N-2,N-1)}=1$
		\item \textcolor{red}{(1P)}  $(23)_7$ (17) + $(16)_7$...
		\item \textcolor{red}{(1P)} $(19)_{h}$  (25) + $(12)_{h}$ (18) = $(2B)_{h}$ (43)
	\end{enumerate}
	\item 
	\begin{enumerate}
		\item Wandeln Sie die folgenden 2er-Komplementzahlen ins Dezimalsystem um.
		\begin{enumerate}
			\item \textcolor{red}{(0.5P)} $(10101010)_{\mathbb{Z}}$ = -128 +42 = -86
			\item \textcolor{red}{(0.5P)} $(11100000)_{\mathbb{Z}}$ = -128 + 96 = -32
		\end{enumerate}
		\item \textcolor{red}{(0.5P)} Übertragsbit = 1  ~~~~~~~~~ \textcolor{red}{(1P)} $\rightarrow$ Ergebnis als vorzeichenlose Zahl ungültig \\
		\textcolor{red}{(0.5P)} Überlaufsbit = $C \oplus c = 1 \oplus 1$ = 0~~~~~~~~ \textcolor{red}{(1P)} $\rightarrow$ Ergebnis als Zweierkomplementzahl gültig 
	\end{enumerate}
	\item \textcolor{red}{(1P)} $x = 7 = (0111)_Z = (000111)_Z$\\
	\textcolor{red}{(1P)} $z = -7 = (1001)_Z = (111001)_Z$
\end{enumerate}
