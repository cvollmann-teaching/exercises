\textbf{Dünnbesetzte Vektoren II}\\
Nehmen Sie an, Sie möchten einen Vektor $x \in \mathbb{R}^n$ als \textit{Dictionary} in der Form \verb|vec[index] = value| speichern.

\begin{enumerate}
	\item Testen und implementieren Sie eine Funktion \verb|dot(vec1, vec2)|, die das Skalarprodukt zweier solcher Vektoren berechnet.
	\item Testen und implementieren Sie eine Funktion \verb|vec3 = sumvec(vec1, vec2)|, die zwei Vektoren addiert und die Summe \verb|sumvecs| zurück gibt. Verändern Sie dabei weder \verb|vec1| noch \verb|vec2|.
	\item Testen und implementieren Sie eine Funktion \verb|vec2 = scalevec(scalar, vec1)|, die einen Vektor mit einem Skalar multipliziert und das Ergebnis zurück gibt. Verändern Sie dabei \verb|vec1| nicht.
\end{enumerate}