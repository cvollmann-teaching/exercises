{\color{solution}
\begin{enumerate}
	\item \begin{enumerate}
		\item 
		\begin{align*}
		A-\lambda I=\begin{pmatrix}1-\lambda&2\\2&1-\lambda\end{pmatrix}&\Rightarrow~~0
		=\text{det}(A-\lambda I)
		= (1-\lambda)^2 - 4~~\textcolor{exampoints}{(1P)}\\
		&\Rightarrow~~\sigma(A)=\{3,-1\}~~\textcolor{exampoints}{(2P)}
		\end{align*}
		\item 
		\begin{align*}
		B-\lambda I=&\begin{pmatrix}-\lambda&0&0\\1&(1-\lambda)&3\\2&0&(1-\lambda)\end{pmatrix}\textcolor{blue}{\begin{matrix}-\lambda&0\\1&(1-\lambda)\\2&0\end{matrix}}\\
		&\\
		&\Rightarrow~~0
		=\text{det}(E-\lambda I)
		=-\lambda(1-\lambda)^2~~\textcolor{exampoints}{(1P)}\\
		&\Rightarrow~~\sigma(B)=\{0,1\}~~\textcolor{exampoints}{( 2P)}
		\end{align*}
		\item \textcolor{exampoints}{(1P)} Triangular matrix\\
	 \textcolor{exampoints}{(1P)}	$ \sigma(C) = diag(C) = \{\pi, -7, i, 0, 0.5\}$
	\end{enumerate}
\item  \textcolor{exampoints}{(2P)} Power Method/Iteration
\item \textcolor{exampoints}{(2P)} yes
 %\textcolor{exampoints}{(1P)} pseudocode: Let $A \in \mathbb{R}^{n\times n}$. $A_0 := A$, for $1\leq k\leq \infty$ do $A_{k }= Q_kR_k$ and $A_{k+1 }= R_kQ_k$; done.
\end{enumerate}
}
