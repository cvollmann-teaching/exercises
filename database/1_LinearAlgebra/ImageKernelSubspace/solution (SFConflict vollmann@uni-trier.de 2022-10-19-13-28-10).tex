% !TeX spellcheck = en_US
{\color{solution}
\begin{enumerate}
	\item 	
	\begin{enumerate}
		\item 	\underline{To show:} $\text{ker}(A)\subset\mathbb{F}^n$ subspace 
		\item[] 	\underline{Proof:}
		\begin{enumerate}
			\item $A\cdot0=0$, so that $0\in\text{ker}(A)$, thus $\text{ker}(A)\neq \emptyset$.
			\item For $i=1,2$ let $\lambda_i\in\mathbb{F},~v_i\in\text{ker}(A),~\text{then by linearity}~
			A(\lambda_1v_1+ \lambda_2v_2)
			=\lambda_1\underbrace{Av_1}_{\textcolor{blue}{=0}}+\lambda_2\underbrace{Av_2}_{\textcolor{blue}{=0}}=0\\
			\Rightarrow~~\lambda_1v_1+ \lambda_2v_2\in\text{ker}(A)$
		\end{enumerate}
		\item \underline{To show:} $\text{Im}(A)\subset\mathbb{F}^m$ subspace
		\item[] 	\underline{Proof:}
		\begin{enumerate}
			\item $A\cdot0=0\in\text{Im}(A)$, thus nonempty.
			\item For $i=1,2$ let $\lambda_i\in\mathbb{F},~w_i\in\text{Im}(A)$, then
			\begin{align*}
			~~&\exists v_1,v_2\in \mathbb{F}^n:~w_1=Av_1,~w_2=Av_2\\
			\Rightarrow~~&\lambda_1 w_1+ \lambda_2 w_2=\lambda_1 Av_1+\lambda_2Av_2=A(\lambda_1 v_1+\lambda_2 v_2)\\
			\Rightarrow~~&\lambda_1 w_1+ \lambda_2w_2\in\text{Im}(A)
			\end{align*}
		\end{enumerate}
		\item[] Example/Remark:
				Note that we always have $0, a_1, a_2 \in \im(A)$\\
	\begin{align*}
		&x=\begin{pmatrix}0\\0\end{pmatrix}~~
	\Rightarrow~~A\begin{pmatrix}0\\0\end{pmatrix}
	=0\cdot\begin{matrix}|\\a_1\\|\end{matrix}+0\cdot\begin{matrix}|\\a_2\\|\end{matrix}
	=\begin{pmatrix}0\\0\end{pmatrix}\in\im(A)\\
	&x=\begin{pmatrix}1\\1\end{pmatrix}~~
	\Rightarrow~~A\begin{pmatrix}1\\1\end{pmatrix}
	=1\cdot\begin{matrix}|\\a_1\\|\end{matrix}+1\cdot\begin{matrix}|\\a_2\\|\end{matrix}
	=a_1+a_2\in\im(A)\\
	&x=\begin{pmatrix}1\\0\end{pmatrix}~~
	\Rightarrow~~Ax=a_1\in\im(A)\\
	&x=\begin{pmatrix}0\\1\end{pmatrix}~~
	\Rightarrow~~Ax=a_2\in\im(A)\\
	\end{align*}
	In general $e_j:=\begin{pmatrix}0\\\vdots\\1\\\vdots\\0\end{pmatrix}\leftarrow \text{j-th}~~\in\mathbb{F}^n$, then $Ae_j=a_j$.
\end{enumerate}
\item Examples:
\begin{enumerate}
	\item Consider the matrix composed of ones from the previous exercise.
	\item Let $A = \begin{pmatrix}
	1&1\\0&2
	\end{pmatrix}$.
	Then $Ax = 0$ implies $x=0$, so that $\text{ker}(A)=\{0\}$. In particular, the columns are linearly independent, so that they form a basis of $\mathbb{R}^2$, with other words: $\mathbb{R}^2 = \text{span}(a_1,a_2)=\text{Im}(A)$.
\end{enumerate}
\end{enumerate}
}
