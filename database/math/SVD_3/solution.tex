{\color{solution}
$A=\begin{pmatrix}\frac{1}{\sqrt{2}}&-\frac{1}{\sqrt{2}}\\1&1\end{pmatrix}$
\begin{align*}
\textcolor{blue}{\text{SVD-Recipe:}}~&\textcolor{blue}{\lambda\in\sigma(A^TA),~\lambda\neq 0,~\tilde{v}~\text{eigenvector}}\\&\textcolor{blue}{(\text{i})~\sigma:=\sqrt{\lambda}}\\&\textcolor{blue}{(\text{ii})~v:=\frac{\tilde{v}}{\|\tilde{v}\|}}\\&\textcolor{blue}{(\text{iii})~u:=\frac{1}{\sigma}Av}
\end{align*}
\begin{enumerate}
	\item 
	Compute SVD and test:%
	%
	\begin{itemize}
		\item $A^TA =\begin{pmatrix}\frac{3}{2}&\frac{1}{2}\\\frac{1}{2}&\frac{3}{2}\end{pmatrix}$
		\item Eigenvalues:\\$\text{det}(A^TA-\lambda I) =(\frac{3}{2}-\lambda)^2-\frac{1}{4}=0~~\Leftrightarrow~~\lambda\in\{1,2\}$
		\item Eigenvectors:
		%
		\begin{enumerate}
			\item \begin{align*}(A^TA-\underbrace{\lambda_1}_{\textcolor{blue}{=2}}I)\tilde{v}=0~~&\Leftrightarrow~~\begin{pmatrix}-\frac{1}{2}&\frac{1}{2}\\\frac{1}{2}&-\frac{1}{2}\end{pmatrix}\tilde{v}=0\\
			&\Leftrightarrow~~\tilde{v}_1=\begin{pmatrix}1\\1\end{pmatrix}
			\end{align*}
			\item 
			\begin{align*}
			(A^TA-\underbrace{\lambda_2}_{\textcolor{blue}{=1}}I)\tilde{v}=0~~&\Leftrightarrow~~\begin{pmatrix}\frac{1}{2}&\frac{1}{2}\\\frac{1}{2}&\frac{1}{2}\end{pmatrix}\tilde{v}=0\\
			&\Leftrightarrow~~\tilde{v}_2=\begin{pmatrix}1\\-1\end{pmatrix}
			\end{align*}
		\end{enumerate}
	%
		\item $\sigma_1:=\sqrt{\lambda_1}=\sqrt{2},~\sigma_2:=\sqrt{\lambda_2}=1,\\
		v_1:=\frac{1}{\sqrt{2}}\begin{pmatrix}1\\1\end{pmatrix},~v_2:=\frac{1}{\sqrt{2}}\begin{pmatrix}1\\-1\end{pmatrix},\\
		u_1:=\frac{1}{\sigma_1}Av_1=\frac{1}{\sqrt{2}}\frac{1}{\sqrt{2}}
		\begin{pmatrix}
		\frac{1}{\sqrt{2}}&-\frac{1}{\sqrt{2}}\\
		1&1
		\end{pmatrix}\begin{pmatrix}1\\1\end{pmatrix}=\begin{pmatrix}0\\1\end{pmatrix},\\
		u_2:=\frac{1}{\sigma_2}Av_2=\frac{1}{\sqrt{2}}\begin{pmatrix}\frac{2}{\sqrt{2}}\\0\end{pmatrix}=\begin{pmatrix}1\\0\end{pmatrix}$
		\item 
		Thus:
		$$
		\Sigma =\begin{pmatrix}\sqrt{2}&0\\0&1\end{pmatrix},~~
		V=\frac{1}{\sqrt{2}}\begin{pmatrix}1&1\\1&-1\end{pmatrix},~~
		U=\begin{pmatrix}0&1\\1&0\end{pmatrix}
		$$
		\item 
		Test:
		$$
		U\Sigma V^T
		=\frac{1}{\sqrt{2}}\begin{pmatrix}0&1\\1&0\end{pmatrix}\underbrace{\begin{pmatrix}\sqrt{2}&0\\0&1\end{pmatrix}\begin{pmatrix}1&1\\1&-1\end{pmatrix}}_{\textcolor{blue}{=\begin{pmatrix}\sqrt{2}&\sqrt{2}\\1&-1\end{pmatrix}}}
		=\frac{1}{\sqrt{2}}\begin{pmatrix}1&-1\\\sqrt{2}&\sqrt{2}\end{pmatrix}=A~~\checkmark$$
	\end{itemize}
	\item 
	\begin{align*}
	A=\sigma_1u_1v_1^T+\sigma_2u_2v_2^T&=\frac{\sqrt{2}}{\sqrt{2}}\begin{pmatrix}0\\1\end{pmatrix}(1, 1)+\frac{1}{\sqrt{2}}\begin{pmatrix}1\\0\end{pmatrix}(1, -1)\\
	&=\begin{pmatrix}0&0\\1&1\end{pmatrix}+\frac{1}{\sqrt{2}}\begin{pmatrix}1&-1\\0&0\end{pmatrix}~~\checkmark
	\end{align*}
	\item 
	Yes, since $\sigma_1\neq0\neq\sigma_2$.
\end{enumerate}
}