Model: $f(x)=c_0 + c_1x$
\begin{enumerate}
	\item 
	$$
	A=\begin{pmatrix}1&0\\1&1\\1&2\\1&2,5\\1&3\\1&4\end{pmatrix}~~\Rightarrow~~
	\underbrace{A^TA}_{\textcolor{blue}{\text{[always symmetric]}}}=\begin{pmatrix}1&1&1&1&1&1\\0&1&2&2,5&3&4\end{pmatrix}\begin{pmatrix}1&0\\1&1\\1&2\\1&2,5\\1&3\\1&4\end{pmatrix}=\begin{pmatrix}6&12,5\\12,5&36,25\end{pmatrix}
	$$
	\item $A^Ty=\begin{pmatrix}0,83\\-1,76\end{pmatrix},~x:=\begin{pmatrix}c_0\\c_1\end{pmatrix}$\\
	Thus $A^TAx=A^Ty$ is equivalent to:
	\begin{align*}
	&\left(
	\begin{array}{cc|c}6&12,5&0,83\\12,5&36,25&-1,76\end{array}
	\right)
	\textcolor{blue}{\begin{pmatrix}1\\2\end{pmatrix}}\\
	\stackrel{\textcolor{violet}{\text{(I)}\leftrightarrow\text{(II)}}}{\rightsquigarrow}
	%
	&\left(
	\begin{array}{cc|c}12,5&36,25&-1,76\\6&12,5&0,83\end{array}
	\right)
	\textcolor{blue}{\begin{pmatrix}2\\1\end{pmatrix}}\\
	\stackrel{\textcolor{violet}{\text{(II)'}=\text{(II)}-\frac{6}{12,5}\text{(I)}}}{\rightsquigarrow}
	&\left(
	\begin{array}{cc|c}12,5&36,25&-1,76\\\textcolor{red}{0,48}&-4,9&1,67\end{array}
	\right)
	\textcolor{blue}{\begin{pmatrix}2\\1\end{pmatrix}}\\
	\Rightarrow~~&c_0=0,85,~c_1=-0,34
	\end{align*}
\end{enumerate}