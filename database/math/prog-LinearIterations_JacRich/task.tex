\section{Splitting Methods: relax. Richardson, relax. Jacobi, Gauß-Seidel and SOR}

\begin{enumerate}
		\item 	Implement a function 
		$$\verb|x, error, numiter = iter_solve(A, b, x0, method="Jacobi", theta=.1, tol=1e-08, maxiter=50)|$$ 
		which takes as arguments
%		The function shall return an approximate solution of the problem $Ax = b$ after performing $m \in \N$ steps of the linear iteration specified by the parameter \verb|method|.	
		\begin{itemize}
			\item A : a matrix $A \in \mathbb{R}^{n \times n}$
			\item b : a vector $b \in \mathbb{R}^ {n}$
			\item x0 : an initial guess $x^0 \in \mathbb{R}^ {n}$
			\item \texttt{method} : optional parameter to choose between relax. Richardson, weighted Jacobi, Gauß-Seidel and SOR and which is set to \verb|"Jacobi"| by default
			\item theta : relaxation parameter $\theta$ which is set to $0.1$ by default\\ (note: Gauß--Seidel is SOR with \texttt{theta=1.0})
			\item tol : error tolerance as float, which is set to $10^{-8}$ by default 
			\item maxiter : maximum number of iterations, which is set to 50 by default
		\end{itemize}
		and then solves the system $Ax=b$ by applying the specified iterative scheme. It shall then return
		\begin{itemize}
			\item \texttt{x} : list of all iterates $x^k$
			\item \texttt{error} : list containing all residuals $\|Ax^k-b\|_2$
			\item \texttt{numiter} : number of iterations that have been performed
		\end{itemize}	
		The iteration shall break if the residual is tolerably small, i.e., 
		$$\|Ax^k-b\|_2 < \texttt{tol}$$
	    or the maximum number of iterations \texttt{maxiter} has been reached.\\~\\
\textit{Hint:} Implement the element-based formulas for the Jacobi, Gauß-Seidel and SOR method (see previous exercise).
		\item \textbf{2d:} Test \underline{all} methods on the following two-dimensional setting:
		$$A = 4 \begin{pmatrix}
		2&-1\\ -1&2
		\end{pmatrix}, ~~b = \begin{pmatrix}
		0\\0
		\end{pmatrix}.$$ 
		What is the exact solution $x^*$? Play around with the parameters \texttt{x0, theta, tol} and \texttt{maxiter}.
		Also create the following two plots for one fixed setting:
		\begin{itemize}
			\item Plot the error $\|Ax^k - b\|_2$  for each iterate $x^k \in \mathbb{R}^2$, $k=1,\dots,m$, for \underline{all} methods into one plot (use different colors).
			\item Plot the iterates $x^k \in \mathbb{R}^2$, $k=1,\dots,m$, themselves for all methods into one plot (use different colors).
		\end{itemize}
		\item \textbf{nd:} Next, test all methods on the higher--dimensional analogue
		$$
		A = n^2 \left(\begin{array}{rrrrr}                                
		2 & -1  &0   & \hdots   & 0 \\                                               
		-1 &  2 & -1  &    &   \vdots \\                                               
		0&  \ddots &  \ddots &\ddots  &0  \\ 
		\vdots  &    &  -1 &  2 & -1  \\ 
		0 &   \hdots  & 0& -1  &  2 \\
		\end{array}\right)\in \mathbb{R}^{n \times n},$$
		for different dimensions $n\in\N$ and data $b,x^0 \in \mathbb{R}^n$ of your choice. Play around with the parameters.
	\end{enumerate}
	\textit{Hint: } Of course, it can happen that the iterations do not converge. Use small values for $\theta$ when you use the Richardson iteration. This will assure that $\rho(I-NA)<1$. 
