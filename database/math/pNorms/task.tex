\textbf{The $p$-Norm}

{\color{navy}
A mapping $\|\cdot\|: \mathbb{R}^n  \to [0,+\infty)$ is called \textbf{norm} on $\mathbb{R}^n$ if it satisfies the three properties
\begin{itemize}
\item[i)] $\|x\|=0\ \Rightarrow x=0$ \hspace{3cm}\textit{(positive definite/ point separating)}
\item[ii)] $\|r\cdot x\|=|r|\cdot \|x\|\, ,\ \forall x\in\mathbb{R}^n, r\in\mathbb{R}$ \hspace{0.43cm}\textit{(absolutely homogeneous)}
\item[iii)] $\|x+y\|\le \|x\|+\|y\|, \ \forall x,y\in\mathbb{R}^n$ \hspace{0.44cm}\textit{(subadditive/ triangle inequality)}
\end{itemize}
The most common example is given by the \textbf{$p$-norm} for $p \in [1,+\infty)$, defined by
$$\|x\|_p := \left(\sum_{i=1}^n |x_i|^p\right)^\frac{1}{p}\nonumber$$
and the \textbf{supremum (or maximum) norm} defined by
$$\|x\|_\infty := \max_{i= 1,\ldots, n} |x_i|.\nonumber$$
\textit{Remark:} For a fixed $x\in\mathbb{R}^n$ one can show $\lim_{p \to \infty} \|x\|_p = \|x\|_\infty$.\\
}
~\\
\textbf{Tasks:}\\
\begin{enumerate}
\item Draw the sets $\{x \in \mathbb{R}^2: \|x\|_p = 1 \}$ for $p = 1,2,\infty.$
\item Show that the Euclidean norm $\|\cdot\|_2: \mathbb{R}^n  \to [0,+\infty),~ x \mapsto \sqrt{\sum_{i=1}^n x_i^2} = \sqrt{x^\top x}$ satisfies i)-iii).\\
\textit{Hint: } For iii) use the Cauchy-Schwarz inequality from the lecture and show $\|x+y\|^2 \leq (\|y\|_2 + \|x\|_2)^2$ and also use  $|a+b| \leq |a| + |b|$ for $a,b\in\mathbb{R}$ (i.e., the triangle inequality is true on $\mathbb{R}^1$).

\end{enumerate}
