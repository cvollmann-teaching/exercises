% !TeX spellcheck = en_US
\textbf{Properties of Eigenvalues}

%Prove \underline{three} (you get \textbf{2 bonus points} for a fourth solution) of 
Prove the following statements (see also the corresponding Lemma from the lecture):
\begin{enumerate}
	\item  \textit{The eigenvalues of the powers of a matrix:}\\ 
	Let $A \in \F^{n\times n} ,~\lambda \in \sigma(A)$ then $\lambda^k \in \sigma(A^k)~\text{for any}~k\in \N$ .
	\item	\textit{Eigenvalues of invertible Matrices:}\\ Let $A\in \mathbb{F}^{n \times n}$ be invertible and $\lambda \in \sigma(A)$, then $\lambda \neq 0$ and $\frac{1}{\lambda}$ is an eigenvalue of $A^{-1}$.
	\item \textit{Eigenvalues of a scaled matrix:}\\ 
	Let $A\in \mathbb{F}^{n \times n}$ and $\lambda \in \sigma(A)$, then $\alpha\lambda \in \sigma(\alpha A)$ for any $\alpha \in \mathbb{F}$.
	\item \textit{Real symmetric matrices have real eigenvalues:} {\small\color{red}(Not obvious without using properties of complex numbers! Solution not relevant for exam and thus not discussed here.)}\\ $A \in \R^{n\times n},~A = A^T ~~\Rightarrow~~ \sigma(A)\subset\R$.
	\item \textit{The eigenvalues of real orthogonal matrices:} {\small\color{red}(Not obvious without using properties of complex numbers! Solution not relevant for exam and thus not discussed here.)}\\
	$Q \in \Rnn$ be orthogonal, $\lambda = a+ib \in \sigma(Q) ~~\Rightarrow~~|\lambda|:=\sqrt{a^2+b^2}=1$
	\item \textit{The eigenvalues of an upper (or lower) triangular matrix are sitting on its diagonal:}\\ 
	Let $U\in \mathbb{F}^{n \times n}$ with $u_{ij} = 0$ for $i > j$. Then $\sigma(U) = \{u_{11}, \ldots, u_{nn}\}$.
	\item \textit{Similar matrices have the same eigenvalues:}\\ Let $A \in \mathbb{F}^{n \times n}$ and $T \in GL_n(\mathbb{F})$, i.e., $T$ is invertible. Then $\sigma(A) = \sigma(T^{-1}AT)$.
	\item\textit{Eigenvalues of a shifted matrix:}\\ Let $A\in \mathbb{F}^{n \times n}$ and $\lambda \in \sigma(A)$, then $(\lambda - \alpha)$ is an eigenvalue of $(A - \alpha I)$ for any $\alpha \in \mathbb{F}$.
	\item \textit{Symmetric matrices have orthogonal eigenvectors:} {\small\color{red}(Solution not relevant for exam and thus not discussed here.)}\\ Let $\lambda_1 \neq \lambda_2$ be two \textit{distinct} eigenvalues of a real \textit{symmetric} matrix $A \in \mathbb{R}^{n \times n}$ (i.e., $A = A^T$), and let 
$v_1, v_2 \in \mathbb{R}^n$ be corresponding eigenvectors. Proof that $v_1$ and $v_2$ are orthogonal, i.e., $v_1^\top v_2 = 0$.
\end{enumerate}
