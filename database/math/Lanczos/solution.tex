{\color{solution}
\begin{enumerate}
	\item Let $k < j$. Since $q_k^\top q_j = \frac{1}{ \|\widehat{q}_j\|_2} q_k^\top \widehat{q}_j$ it suffices to show that $q_k^\top \widehat{q}_j=0$. Now let $v:=Aq_{j-1}$, then
	\begin{align*}
q_k^\top 	\widehat{q}_j  = q_k^\top \left(v - \sum_{\ell = 1}^{j-1} q_\ell^\top v  \cdot q_\ell\right)  
	&= q_k^\top v - \sum_{\ell = 1}^{j-1} q_\ell^\top v   \cdot \underbrace{q_k^\top q_\ell}_{= \delta_{k\ell}} \\
	&=q_k^\top v -  q_k^\top v\cdot 1\\
	&=0.  
	\end{align*}
	\item By definition of the matrix product we obtain, for $1\leq \ell,j \leq r$,
	\begin{align*}
	  H_r^{\ell k} = (Q_r^\top AQ_r)_{\ell k} = q_\ell ^\top Aq_k  .
	\end{align*}
	These are precisely the projection lengths that are computed during the Arnoldi iteration. Since by definition $Aq_j$ can be uniquely generated by $q_1,\ldots, q_{j+1}$, we have that $h_{ij} = 0$ for all $i> j+1$. In particular, $H_r$ is an upper \textit{Hessenberg} matrix (having precisely one subdiagonal).
	\item If $A$ is symmetric, then $H_r = Q_r^\top A Q_r$ is symmetric, so that it simplifies to a tridiagonal matrix. In particular $h_{ij} = (Aq_j)^\top q_i = 0$ for all $i,j$ with $|i-j|> 2$ and Arnoldi becomes Lanczos by accounting for the simplification
	\begin{align*}
	 \widehat{q}_j = Aq_{j-1} - \sum_{\ell = 1}^{j-1} q_\ell^\top(Aq_{j-1})  \cdot q_\ell
	 = Aq_{j-1} - q_{j-2}^\top(Aq_{j-1})  \cdot q_{j-2} - q_{j-1}^\top(Aq_{j-1})\cdot q_{j-1}.
	\end{align*}
  \item Since $Q_n^TAQ_n \in \mathbb{R}^{n \times n}$ is orthogonally similar to $A$, it has the same eigenvalues as $A$.
\end{enumerate}



}