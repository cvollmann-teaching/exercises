% !TeX spellcheck = en_US
\begin{enumerate}
	\item Give an example for a matrix $A \in \mathbb{R}^{2 \times 2}$ where the LU-decomposition algorithm
	necessarily needs a permutation step. 
\item Let $R=(r_{ij})_{ij} \in \mathbb{R}^{n \times n}$ be a (lower or upper) triangular matrix with $r_{nn} = 0$. Is $R$ invertible? Explain your answer.
\item When is a diagonal matrix invertible? Write down the inverse in this case.
\item Let $A\in\mathbb{R}^{n \times n}$ and $b \in \mathbb{R}^n$. Also let $\widehat{Q}\widehat{R}$ be a $\underline{\text{reduced}}$ $QR$--decomposition of $A$. What properties do $\widehat{Q}$ and $\widehat{R}$ have? How can we use it to solve the system $Ax = b$?
\item Consider the system
$$
\begin{pmatrix}
0	&-2				& 1\\
2	&2				& 0\\
-2	&-\frac{3}{2}	&0
\end{pmatrix} x = 
\begin{pmatrix}
-5\\
6\\
-5
\end{pmatrix}.
$$
\begin{enumerate}
	%	\item Define a matrix $A$ and a vector $b$, so that this system reads as $Ax = b$. 
	\item  Define a matrix $A$ and a vector $b$, so that this system reads as $Ax = b$. Then compute an $LU$-decomposition of $A$ by applying Gaussian elimination with \underline{\textbf{row pivoting}}. Denote the respective matrices $L$, $U$ and $P$, such that $PA = LU$.\\
	(\textit{Hint: Verify the desired properties of the factor matrices and test whether  $PA = LU$ holds.})
	\item Use the result from the $LU$-decomposition to determine an $x$ which solves $Ax=b$.\\
	(\textit{Hint: Test whether $Ax=b$ holds.})
\end{enumerate}
\end{enumerate}
 