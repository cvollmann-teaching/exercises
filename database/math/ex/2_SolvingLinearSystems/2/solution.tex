{\color{solution}
\begin{enumerate}
	\item \textcolor{exampoints}{(2P)}  $x_1 = 6/3 = 2$\\
	\textcolor{exampoints}{(2P)} 
	$x_2 = \frac{1}{2}(3 - 1x_1) =  \frac{1}{2}$
	%
	\item \textcolor{exampoints}{(3P)}
	$\widehat{Q} = \begin{pmatrix}
	 -1&0\\0&1\\0&0
	\end{pmatrix}$\\
 \textcolor{exampoints}{(2P)}	$\widehat{R}= \begin{pmatrix}
	2 & 0\\0&5
\end{pmatrix} $
\item We find
$$\textcolor{exampoints}{(3P)}~P = \begin{pmatrix}
0 & 1\\
1 & 0
\end{pmatrix},~~ \textcolor{exampoints}{(3P)}~L=\begin{pmatrix}
1 & 0\\
\frac{1}{3} & 1
\end{pmatrix},~~ \textcolor{exampoints}{(3P)} ~U=\begin{pmatrix}
3 & 6\\
0 & -1
\end{pmatrix}
$$
\item \textcolor{exampoints}{(1P)} Exactly one solution, which is \\ \textcolor{exampoints}{(1P)} $x=A^{-1}b$
\item \textcolor{exampoints}{(1P)} Yes,  \\
\textcolor{exampoints}{(1P)} because $b \in\mathbb{R}^m=\text{Im}(A)=\{Ax: x\in\mathbb{R}^n\}$, thus $b=Ax$ for some $x$. (surjective)
\item \textcolor{exampoints}{(1P)} No solution, \\
\textcolor{exampoints}{(1P)} because for all  $x\in\mathbb{R}^2$ we have $0^\top x = 0 \neq 4 = b_2$
\end{enumerate}
}
