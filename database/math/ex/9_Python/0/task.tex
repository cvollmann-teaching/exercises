\vspace*{-0.5cm}
\begin{enumerate}
	\item Consider:
\begin{lstlisting}[language=python,numbers=left]
def fun(A,b, m=50):
    n = A.shape[1]
    x = numpy.zeros(n)
    N = 1 / A.diagonal()		
    for k in range(m):
        x = x - N * (A @ x - b)
        return x
\end{lstlisting}
	\begin{enumerate}
		\item Which algorithm is implemented?
		\item What is its purpose? 
		\item Which role does \verb|N| play here?
	\end{enumerate}
\item Consider:
\begin{lstlisting}[language=python,numbers=left]
def fun(A, m=50):
    n = A.shape[1]
    x = numpy.zeros(n)
    x[0] = 1	
    for k in range(m):
        x = A @ x
        x = x / numpy.linalg.norm(x)
    return x
\end{lstlisting}
\begin{enumerate}
	\item Which algorithm is implemented? 
	\item What is the purpose of this algorithm?
\end{enumerate}
\item Consider:
\begin{lstlisting}[language=python,numbers=left]
def fun(A, m=50):
    from scipy.linalg import qr
    for k in range(m):
        Q, R = qr(A)
        A = R @ Q
    return A.diagonal()
\end{lstlisting}
\begin{enumerate}
	\item Which algorithm is implemented? 
	\item What is the purpose of this algorithm?
\end{enumerate}
\end{enumerate}
 