\vspace*{-0.5cm}
\begin{enumerate}
\item Compute the eigenvalues of the following matrices.

\begin{enumerate}
	\item 
	$$
	A = \begin{pmatrix}
	0&-1\\
	1&0
	\end{pmatrix}
	$$
	%
	\item
	$$
	B = \begin{pmatrix}
	0&0&0\\
	1&1&3\\
	2&0&1
	\end{pmatrix}
	$$
	%
	\item 
	$$C=
	\begin{pmatrix}
	\pi& 0& 0&0\\
	1 & -7&  0&0\\
	2 & 0& i   &0\\
	4 & 7& 9& \frac{1}{2}
	\end{pmatrix}
	$$
\end{enumerate}
\item Assume you need to compute the eigenvector corresponding to the largest (in magnitude) eigenvalue. Which algorithm would you use?
\item Let $A\in\mathbb{R}^{n \times n}$ and $A_k$ be the $k$-th iterate of the $QR$--Algorithm. Is it true that $A$ and $A_k$ have the same spectrum? [yes/no]% Write down a pseudocode of it.
%\item ---
%	\item Let $\lambda_1 \neq \lambda_2$ be two eigenvalues of a \textit{symmetric} matrix $A \in \mathbb{R}^{n \times n}$, and let 
%	$v_1, v_2 \in \mathbb{R}^n$ be corresponding eigenvectors. Proof that $v_1^\top v_2 = 0$. 
%	 \item Let $T \in \mathbb{R}^{n \times n}$ be an invertible matrix and $A \in \mathbb{R}^{n \times n}$ be any matrix. Show that $A$ and $T A T^{-1}$ have the same eigenvalues.
%
%
%\item You need to compute the eigenvector corresponding to the largest eigenvalue in magnitude. Which algorithm would you use?
%\item What is the purpose of the $QR$ Algorithm? Write down its iteration instruction.
%\item Let $A \in \mathbb{R}^{n \times n}$ be symmetric (implying real eigenvalues) and \textit{negative} definite matrix, i.e., $x^\top A x < 0$ for all $x \in \mathbb{R}^{n}\setminus \{0\}$. Show that the eigenvalues of $A$ are strictly negative, i.e., $\lambda < 0 $ for all $\lambda \in \sigma(A) \subset \mathbb{R}$.
%
%\item Compute the eigenvalues of the following matrices.
%\begin{enumerate}
%	\item $$ A= 
%	\begin{pmatrix}
%	0&\frac{1}{2}\\
%	\frac{1}{2}&0
%	\end{pmatrix}
%	$$
%	%
%	\item
%	$$ B=
%	\begin{pmatrix}
%	1&0&0\\
%	3&1&4\\
%	0&1&1
%	\end{pmatrix}
%	$$
%	\item $$C=
%	\begin{pmatrix}
%	\pi& 3& -1& 2\\
%	0 & 4& 1  & 7\\
%	0 & 0& i  & 0\\
%	0 & 0& 0  & \frac{1}{2}
%	\end{pmatrix}
%	$$
%	
%\end{enumerate}
%\item Compute the eigenvalues of the following matrices.
%
%\begin{enumerate}
%	\item 
%	$$
%	A = \begin{pmatrix}
%	1&2\\
%	2&1
%	\end{pmatrix}
%	$$
%	%
%	\item
%	$$
%	B = \begin{pmatrix}
%	0&0&0\\
%	1&1&3\\
%	2&0&1
%	\end{pmatrix}
%	$$
%	%
%	\item 
%	$$C=
%	\begin{pmatrix}
%	\pi& 0& 0& 0 &0\\
%	1 & -7&  0& 0 &0\\
%	2 & 0& i  & 0 &0\\
%	3 & 0& 8  & 0 & 0\\
%	4 & 7& 9& 0& \frac{1}{2}
%	\end{pmatrix}
%	$$
%\end{enumerate}
%\item Compute the eigenvalues of the following matrices.
%\begin{enumerate}
%	\item $$ A= 
%	\begin{pmatrix}
%	0&\frac{1}{2}\\
%	\frac{1}{2}&0
%	\end{pmatrix}
%	$$
%	%
%	\item
%	$$ B=
%	\begin{pmatrix}
%	0&0&0\\
%	1&0&-9\\
%	0&1&6
%	\end{pmatrix}
%	$$
%	%		\item $$C=
%	%	\begin{pmatrix}
%	%	\pi& 3& -1& 6\\
%	%	0 & 4& 1  & 7\\
%	%	0 & 0& i  & 0\\
%	%	0 & 0& 0  & \frac{1}{2}
%	%	\end{pmatrix}
%	%	$$
%\end{enumerate}
%
%% exam-EigenDecomposition
%\item Let $$A:=\begin{pmatrix}2&3\\3&2 \end{pmatrix} \in \mathbb{R}^{2\times 2}. $$
%\begin{enumerate}
%	\item Why do we know that an eigendecomposition $Q \Lambda Q^T$ of $A$ exists?
%	\item What properties do $Q$ and $\Lambda$ have, if $Q \Lambda Q^T$ is an eigendecomposition of $A$?
%	\item Find such matrices $\Lambda$ and $Q$.
%\end{enumerate}
%\textit{Hint:} Compute $Q \Lambda Q^T = A$ to check your result.
%\item Let $A \in \mathbb{F}^{n \times n}$ be invertible and $\lambda$ some eigenvalue of $A$.
%\begin{enumerate}
%	\item Please show that $\lambda \neq 0$.
%	\item Please show that $\frac{1}{\lambda}$ is an Eigenvalue of $A^{-1}$.
%\end{enumerate}
\end{enumerate}
 