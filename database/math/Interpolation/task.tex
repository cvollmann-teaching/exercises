\section{Interpolation}
Let $(x_1,y_1), (x_2,y_2)$ and $(x_3,y_3)$ be points in the real plain $\mathbb{R}^2$. If $x_i \neq x_j$ for $i \neq j$ there is exactly one 
parabola $f\colon \mathbb{R} \to \mathbb{R}$ (i.e., there are uniquely determined coefficients $a,b,c \in \mathbb{R}$) which satisfies
\begin{align}
f(x_i) = ax_i^2 + b x_i + c = y_i \label{eq} ~~~~~~\text{for all $i=1,2,3$.}
\end{align}

\begin{enumerate}
	\item Please implement a function \verb|interpolate(data)| which takes the points $(x_i, y_i)$ for $i=1,2,3$ as arguments and returns
	the coefficients $a,b,c$, for which \eqref{eq} holds.
	\item Implement a function \verb|parabola(x, coeff)| which evaluates a parabola depending on an argument $x$ and some coefficients $a,b,c$. Then randomly generate some data points $(x_i, y_i)$ for $i=1,2,3$, determine the corresponding coefficients $a,b,c$ with \verb|interpolate(data)| and create a plot which contains the pairs $(x_i, y_i)$ for $i=1,2,3$ and the associated fitted function $f$ from \eqref{eq} with the determined coefficients.
	\item Assume the measuring points $x_1, x_2$ and $x_3$ are fixed and we want to obtain many interpolations for different values of $y_1, y_2, y_3$. What could be changed in your implementation to possibly make it more efficient?
\end{enumerate}
\textit{Hint: } Interpret task (i) as linear system of equations and solve it with \verb|linalg.solve|. 
%If you are already familiar with the definition of classes in Python you could use an appropriate class definition to solve (iii).