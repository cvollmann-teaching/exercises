\section{Least Squares}
\begin{table}[h]
	\centering
	\begin{tabular}{l|c c c c c c}
		x&0&1&2&2.5&3&4\\
		\hline
		y&0.15&1&0.84&0&-0.2&-0.96
	\end{tabular}
	\caption{This table contains a dataset of pairs $(x_i, y_i)$ for $i = 1,\dots,6$ which is used in exercise \ref{ex:lstsq} and \ref{ex:proglstsq}.}
	\label{tab:data}
\end{table}
\label{ex:proglstsq}
Let the data points $(x_i, y_i)$ for $i=1,\dots,6$ be given as in Table \ref{tab:data}.
\begin{enumerate}
	\item Solve the least squares problem of Exercise \ref{ex:lstsq} in Python using the \verb|lstsq()| function of 
	\verb|scipy.linalg| as presented in the lecture.
	\item Plot the data points and the fitted line.
	\item You are given the additional information that the data stems from a sine-function of the form $ \sin(x)a + b \approx y$, i.e. the adjusted problem then is of the form
	\begin{align*}
	\min \limits_{a,b} \sum_{i=1}^6 ( \sin(x_i)a + b - y_i)^2.
	\end{align*}
	Solve the adjusted problem with the \verb|lstsq()| function.
	\item Plot the data points and the fitted curve.
\end{enumerate}
