{\color{solution}
You can later use your Python Code to check your $LU$ decomposition $PA=LU$ with all intermediate steps. Therefore we just copy the factors and solution here.
\begin{enumerate}
	\item Unique solution: 
		$$P = \begin{pmatrix}
0&0&1\\1&0&0\\0&1&0
	\end{pmatrix} $$
		$$L = \begin{pmatrix}
	1&0&0\\
	2/3&1&0\\
	1/3&-1/7&1
	\end{pmatrix} $$
		$$U = \begin{pmatrix}
	3&-2&2\\
	0&7/3&5/3\\
	0&0&-10/7
	\end{pmatrix} $$
	Solving steps yields: $x = (1,-2,-1)^T$, thus $S=\{(1,-2,-1)^T\}$
	\item Infinitely many solutions: The algorithm outputs the following arrays
			$$P = \begin{pmatrix}
	0&0&1\\1&0&0\\0&1&0
	\end{pmatrix} $$
	$$L = \begin{pmatrix}
	1&0&0\\
	1/3&1&0\\
	2/3&-1&1
	\end{pmatrix} $$
		$$U = \begin{pmatrix}
		3&2&3\\
		0&4/3&1\\
		0&0&0
		\end{pmatrix} $$
		and we obtain
		$$z:= L^{-1} P^\top b = \begin{pmatrix}
		4\\ -1/3\\0
		\end{pmatrix}. $$
		Therefore we see that the system $Ux = z$ has infinitely many solutions (last zero row in $U$ and zero in same row in $z$). By solving $Ux = z$ we find
				$$
				\textcolor{violet}{\begin{matrix}(\text{II})\\ (\text{I})\\\  \end{matrix}}
				\begin{array}{l}
				\Rightarrow\ \ 4/3x_2 +1x_3 =  -1/3\ \ \Rightarrow\ \ x_2 = -\frac{1}{4}(1+3x_3)\\
				\Rightarrow\ \ 3x_1+2x_2+3x_3=4\ \ \Rightarrow\ \ x_1 =\frac{1}{3}(4-2x_2-3x_3)=\frac{3}{2} - \frac{1}{2}x_3				
				\end{array}
				$$
				\begin{align*}
				\Rightarrow\ \ S&:=\{x\in\mathbb{R}^3: Ax=b\}\\
				&=\{x\in\mathbb{R}^3:x_1=\tfrac{1}{2}(3 - x_3), ~~x_2= -\tfrac{1}{4}(1+3x_3), ~~x_3\in\mathbb{R}\}\\
				&\stackrel{\textcolor{blue}{s:=x_3\in\mathbb{R}^3}}{=}\begin{Bmatrix} \begin{pmatrix}\frac{3}{2}\\-\frac{1}{4}\\0\end{pmatrix}+s\begin{pmatrix}-\frac{1}{2}\\-\frac{3}{4}\\1\end{pmatrix}:s\in\mathbb{R}\end{Bmatrix},
				\text{(i.e., we have infinitely many solutions!)}
				\end{align*}
	\item No solution: $S = \emptyset$\\

			$$P = \begin{pmatrix}
	0&0&1\\1&0&0\\0&1&0
	\end{pmatrix} $$
	$$L = \begin{pmatrix}
	1&0&0\\
	1/2&1&0\\
	1/2&-1&1
	\end{pmatrix} $$
	$$U = \begin{pmatrix}
2&0&2\\
	0&-1&-1\\
	0&0&0
	\end{pmatrix} $$
			and we obtain
	$$z:= L^{-1} P^\top b = \begin{pmatrix}
	1\\ -1/2\\ 1
	\end{pmatrix}. $$
	Thus, last row in $U$ is a zero row but $1\neq 0$ in $z$.
\end{enumerate}

%\textbf{Note that the pivoting strategy may deviate from the algorithm presented in the lecture.} In fact, for ease of computation I have chosen rather ``1'' as pivot than the element with largest magnitude. Consequently you may have derived different factors $L$, $U$ and $P$. However, important is that they have the correct form and that $PA =LU$.
%\begin{enumerate}
%	\item 
%	\begin{itemize}
%		\item[1)] 
%		\underline{FACTORIZATION:}
%		\begin{align*}
%		&\left(
%		\begin{array}{ccc|c} 2&1&3&-3\\1&-1&-1&4\\3&-2&2&5\end{array}
%		\right)
%		\textcolor{violet}{\begin{matrix}(\text{I})\\ (\text{II})\\ (\text{III})\end{matrix}}\textcolor{blue}{\begin{pmatrix}1\\2\\3\end{pmatrix}}\\
%		\stackrel{\textcolor{violet}{(\text{II})\leftrightarrow(\text{I})}}{\rightsquigarrow}
%%
%		&\left(
%		\begin{array}{ccc|c} 1&-1&-1&4\\2&1&3&-3\\3&-2&2&5\end{array}
%		\right)
%		\textcolor{violet}{\begin{matrix}(\text{I})\\ (\text{II})\\ (\text{III})\end{matrix}}\textcolor{blue}{\begin{pmatrix}2\\1\\3\end{pmatrix}}\\
%		\stackrel{\textcolor{violet}{(\text{II})'=(\text{II})-2(\text{I})}}{\rightsquigarrow}
%%		
%		&\left(
%		\begin{array}{ccc|c} 1&-1&-1&4\\\textcolor{red}{2}&3&5&-11\\3&-2&2&5\end{array}
%		\right)
%		\textcolor{violet}{\begin{matrix}(\text{I})\\ (\text{II})'\\ (\text{III})\end{matrix}}\textcolor{blue}{\begin{pmatrix}2\\1\\3\end{pmatrix}}\\
%		\stackrel{\textcolor{violet}{(\text{III})'=(\text{III})-3(\text{I})}}{\rightsquigarrow}
%%		
%		&\left(
%		\begin{array}{ccc|c} 1&-1&-1&4\\\textcolor{red}{2}&3&5&-11\\\textcolor{red}{3}&1&5&-7\end{array}
%		\right)
%		\textcolor{violet}{\begin{matrix}(\text{I})\\ (\text{II})'\\ (\text{III})'\end{matrix}}\textcolor{blue}{\begin{pmatrix}2\\1\\3\end{pmatrix}}\\
%		\stackrel{\textcolor{violet}{(\text{II})'\leftrightarrow(\text{III})'}}{\rightsquigarrow}
%%		
%		&\left(
%		\begin{array}{ccc|c} 1&-1&-1&4\\\textcolor{red}{3}&1&5&-7\\\textcolor{red}{2}&3&5&-11\end{array}
%		\right)
%		\textcolor{violet}{\begin{matrix}(\text{I})\\ (\text{II})'\\ (\text{III})'\end{matrix}}\textcolor{blue}{\begin{pmatrix}2\\3\\1\end{pmatrix}}\\
%		\stackrel{\textcolor{violet}{(\text{III})''=(\text{III})'-3(\text{II})'}}{\rightsquigarrow}
%%		
%		&\left(
%		\begin{array}{ccc|c} 1&-1&-1&4\\\textcolor{red}{3}&1&5&-7\\\textcolor{red}{2}&\textcolor{red}{3}&-10&10\end{array}
%		\right)
%		\textcolor{violet}{\begin{matrix}(\text{I})\\ (\text{II})'\\ (\text{III})''\end{matrix}}\textcolor{blue}{\begin{pmatrix}2\\3\\1\end{pmatrix}}
%		\end{align*}
%		$$
%		\Rightarrow~~P = \begin{pmatrix}0&1&0\\0&0&1\\1&0&0\end{pmatrix},~~
%		L = \begin{pmatrix}1&0&0\\3&1&0\\2&3&1\end{pmatrix},~~ U=\begin{pmatrix}1&-1&-1\\0&1&5\\0&0&-10\end{pmatrix}.
%		$$
%		\item[2)] \underline{SOLUTION:}
%		\begin{enumerate}
%			\item 
%			\underline{1. Possibility:} Solving final tableau, i.e., $Ux = L^{-1}Pb=\begin{pmatrix}4\\-7\\10\end{pmatrix}$ 
%			$$
%			\textcolor{violet}{\begin{matrix}(\text{III})''\\ (\text{II})'\\ (\text{I})\end{matrix}}
%			\begin{array}{l}\Rightarrow\ \ -10x_3 = 10\ \ \Rightarrow \ \ x_3=-1\\
%			\Rightarrow\ \ x_2+5x_3 = -7\ \ \Rightarrow\ \ x_2+5(-1) = -7\ \ \Rightarrow\ \ x_2=-2\\
%			\Rightarrow\ \ x_1-x_2-x_3 =4\ \ \Rightarrow\ \ x_1+2+1=4\ \ \Rightarrow\ \ x_1=1\end{array}
%			$$
%			\item 
%			\underline{2. Possibility: (equivalent)}\\ Assume we were just given $L,U$ and $P$, then we could solve the system in two steps:\\ 
%			i) Solve $ Lz=Pb=\begin{pmatrix}4\\5\\-3\end{pmatrix}$ for $z$:\\~\\
%							$\begin{pmatrix} 1&0&0\\3&1&0\\2&3&1\end{pmatrix}\begin{pmatrix}z_1\\z_2\\z_3\end{pmatrix}=\begin{pmatrix}4\\5\\-3\end{pmatrix}\ \ \ 
%			\begin{array}{l}
%			\Rightarrow\ \ z_1 = 4\\\Rightarrow\ \ 3z_1+z_2=5\ \ \Rightarrow\ \ z_2=-7\\\Rightarrow\ \ 2z_1+3z_2+z-3=-3\ \ \Rightarrow\ \ z_3=10
%			\end{array}$\\~\\
%			%
%			ii) Solve $Ux= z$ for $x$:\\~\\
%			 $\begin{pmatrix}1&-1&-1\\0&1&5\\0&0&-10\end{pmatrix}\begin{pmatrix}x_1\\x_2\\x_3\end{pmatrix}=\begin{pmatrix}4\\-7\\10\end{pmatrix}\ \ \ \Rightarrow\ \ x=\begin{pmatrix}-1\\-2\\1\end{pmatrix}$
%	\end{enumerate}
%	\end{itemize}
%	\item 
%	\begin{itemize}
%		\item[1)] \underline{FACTORIZATION:}
%		\begin{align*}
%		&\left(
%		\begin{array}{ccc|c} 1&2&2&1\\2&0&1&3\\3&2&3&4\end{array}
%		\right)
%		\textcolor{violet}{\begin{matrix}(\text{I})\\ (\text{II})\\ (\text{III})\end{matrix}}\textcolor{blue}{\begin{pmatrix}1\\2\\3\end{pmatrix}}\\
%		\stackrel{\textcolor{violet}{(\text{II})'=(\text{II})-2(\text{I})}}{\rightsquigarrow}
%%		
%		&\left(
%		\begin{array}{ccc|c}1&2&2&1\\\textcolor{red}{2}&-4&-3&1\\3&2&3&4\end{array}
%		\right)
%		\textcolor{violet}{\begin{matrix}(\text{I})\\ (\text{II})'\\ (\text{III})\end{matrix}}\textcolor{blue}{\begin{pmatrix}1\\2\\3\end{pmatrix}}
%		\\
%		\stackrel{\textcolor{violet}{(\text{III})'=(\text{III})-3(\text{I})}}{\rightsquigarrow}
%%		
%		&\left(
%		\begin{array}{ccc|c}1&2&2&1\\\textcolor{red}{2}&-4&-3&1\\\textcolor{red}{3}&-4&-3&1\end{array}
%		\right)
%		\textcolor{violet}{\begin{matrix}(\text{I})\\ (\text{II})'\\ (\text{III})'\end{matrix}}\textcolor{blue}{\begin{pmatrix}1\\2\\3\end{pmatrix}}\\
%		\stackrel{\textcolor{violet}{(\text{III})''=(\text{III})'-(\text{II})'}}{\rightsquigarrow}
%%		
%		&\left(
%		\begin{array}{ccc|c} 1&2&2&1\\\textcolor{red}{2}&-4&-3&1\\\textcolor{red}{3}&\textcolor{red}{1}&0&0\end{array}
%		\right)
%		\textcolor{violet}{\begin{matrix}(\text{I})\\ (\text{II})'\\ (\text{III})''\end{matrix}}\textcolor{blue}{\begin{pmatrix}1\\2\\3\end{pmatrix}}
%		\end{align*}$$
%		\Rightarrow\ \ P=I_3,\ \ L=\begin{pmatrix}1&0&0\\2&1&0\\3&1&1\end{pmatrix},\ \ U=\begin{pmatrix}1&2&2\\0&-4&-3\\0&0&\textcolor{red}{0}\end{pmatrix}
%		$$
%		$\textcolor{red}{\rightarrow u_{3,3}=0\Rightarrow\  \text{U \underline{not} invertible, but}\ z_3=0\ \Rightarrow\ \text{solution(s) exist}}$
%		\item[2)] 
%		\underline{SOLUTION:}
%		$$
%		\textcolor{violet}{\begin{matrix}(\text{II})\\ (\text{I})\\\  \end{matrix}}
%		\begin{array}{l}
%		\Rightarrow\ \ -4x_2-3x_3 = 1\ \ \Rightarrow\ \ x_2 = \frac{1}{4}(1+3x_3)\\\Rightarrow\ \ x_1+2x_2+2x_3=1\ \ \Rightarrow\ \ x_1 =1-2x_2-2x_3\\=1+\frac{1}{2}+\frac{3}{2}x_3-2x_3=\frac{3}{2}-\frac{1}{2}x_3
%		\end{array}
%		$$
%		\begin{align*}
%		\Rightarrow\ \ S&:=\{x\in\mathbb{R}^3: Ax=b\}\\
%		&=\{x\in\mathbb{R}^3:x_1=\frac{3}{2}-\frac{1}{2}x_3, x_2=-\frac{1}{4}-\frac{3}{4}x_3, x_3\in\mathbb{R}\}\\
%		&\stackrel{\textcolor{blue}{s:=x_3\in\mathbb{R}^3}}{=}\begin{Bmatrix} \begin{pmatrix}\frac{3}{2}\\-\frac{1}{4}\\0\end{pmatrix}+s\begin{pmatrix}-\frac{1}{2}\\-\frac{3}{4}\\1\end{pmatrix}:s\in\mathbb{R}\end{Bmatrix},
%		\text{(i.e., we have infinitely many solutions!)}
%		\end{align*}
%	\end{itemize}
%	\item 
%	\begin{itemize}
%		\item[1)] \underline{FACTORIZATION:}
%	\begin{align*}
%	&\left(
%	\begin{array}{ccc|c} 1&1&2&2\\1&-1&0&0\\2&0&2&1\end{array}
%	\right)
%	\textcolor{violet}{\begin{matrix}(\text{I})\\ (\text{II})\\ (\text{III})\end{matrix}}\textcolor{blue}{\begin{pmatrix}1\\2\\3\end{pmatrix}}\\
%	\stackrel{\textcolor{violet}{(\text{II})'=(\text{II})-(\text{I})}}{\rightsquigarrow}
%%	
%	&\left(
%	\begin{array}{ccc|c}1&1&2&2\\\textcolor{red}{1}&-2&-2&-2\\2&0&2&1\end{array}
%	\right)
%	\textcolor{violet}{\begin{matrix}(\text{I})\\ (\text{II})'\\ (\text{III})\end{matrix}}\textcolor{blue}{\begin{pmatrix}1\\2\\3\end{pmatrix}}\\
%	\stackrel{\textcolor{violet}{(\text{III})'=(\text{III})-2(\text{I})}}{\rightsquigarrow}
%%	
%	&\left(
%	\begin{array}{ccc|c}1&1&2&2\\\textcolor{red}{1}&-2&-2&-2\\\textcolor{red}{2}&-2&-2&-3\end{array}
%	\right)
%	\textcolor{violet}{\begin{matrix}(\text{I})\\ (\text{II})'\\ (\text{III})'\end{matrix}}\textcolor{blue}{\begin{pmatrix}1\\2\\3\end{pmatrix}}\\
%	\stackrel{\textcolor{violet}{(\text{III})''=(\text{III})'-(\text{II})'}}{\rightsquigarrow}
%%	
%	&\left(
%	\begin{array}{ccc|c} 1&1&2&2\\\textcolor{red}{1}&-2&-2&-2\\\textcolor{red}{2}&\textcolor{red}{1}&0&-1\end{array}
%	\right)
%	\textcolor{violet}{\begin{matrix}(\text{I})\\ (\text{II})'\\ (\text{III})''\end{matrix}}\textcolor{blue}{\begin{pmatrix}1\\2\\3\end{pmatrix}}
%	\end{align*}
%	\item[2)] 
%	\underline{SOLUTION:}\\
%	$\textcolor{violet}{(\text{III})''}\ \ \Rightarrow\ \ 0x_1+0x_2+0x_3=-1\ \ \Rightarrow$ no solution\\
%\end{itemize}
%	$\Rightarrow\ \ 
%	P=I_3,~~L=\begin{pmatrix}1&0&0\\{1}&1&0\\{2}&{1}&1\end{pmatrix},~~ U=\begin{pmatrix}1&1&2\\0&-2&-2\\0&0&\textcolor{red}{0}\end{pmatrix}\\
%	\textcolor{red}{\Rightarrow\  \text{U \underline{not} invertible and}\ z_3=-1\neq0\ \Rightarrow\ |S|=0}$
%\end{enumerate}
}