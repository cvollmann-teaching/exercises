% !TeX spellcheck = en_US
\textbf{Solving Linear Systems using $LU$ Decomposition}
%\textit{\color{red} More info during next class (May 18).}

Consider the following linear systems.
\begin{enumerate}
	\item 		\begin{align*}
	&2 x_1 + x_2 + 3x_3 = -3\\
	&x_1 - x_2 - x_3 = 4\\
	&3x_1 - 2x_2 + 2x_3 = 5
	\end{align*}
	\item \begin{align*}
	&x_1 + 2 x_2 + 2x_3  = 1\\
	&2x_1 + x_3 = 3\\
	&3x_1 + 2x_2 + 3x_3 = 4
	\end{align*}
	\item \begin{align*}
			&x_1 +  x_2 + 2x_3  = 2\\
			&1x_1 -x_2 = 0\\
			&2x_1 + 2x_3 = 1
\end{align*}
\end{enumerate}
Cast them into the form $Ax = b$ and compute the $LU$-decomposition of $A$ by \underline{rigorously} applying Algorithm \ref{LUwithRowPiv} (i.e., use the pivot element determined in Line \ref{pivot}):
\begin{itemize}
	\item  Find the matrices $L$, $U$ and $P$ such that $PA=LU$.
	\item Then determine the solution set $S:= \{x\in \mathbb{R}^n:Ax =b  \}$.
\end{itemize}
\textit{Hint: } System 2. does not have a \textit{unique} solution (but infinitely many). Determine the set of vectors $x\in \mathbb{R}^3$ for which the linear system 2. is valid.
\begin{center}
	\hspace*{3cm}\begin{algorithm}
		\textbf{INPUT:} $A\in \mathbb{R}^{n \times n}$ (and $b \in \mathbb{R}^n$)\\
		\textbf{OUTPUT:} LU decomposition $PA = LU$ (and if  $Ax = b$ is uniquely solvable the solution $x\in \mathbb{R}^n$)\\~\\
		{\color{gray}\texttt{\# \textbf{FACTORIZATION}}}\\
		initialize \texttt{piv = [1,2,...,n]}\\
		\For{$j = 1,...,n-1$}{
			{\color{gray}\texttt{\#  \textbf{Find the j-th pivot}}}\\
			$k_j := \arg \max_{k\geq j} |a_{kj}|$ \label{pivot}\\
			\If{$a_{k_jj}\neq 0$}{
				{\color{gray}\texttt{\# \textbf{Swap rows}}}\\
				A[$k_j$,:] $\leftrightarrow$ A[j,:]\\
				{\color{gray}\texttt{\#  by hand we also transform b on the fly}}\\
				b[$k_j$] $\leftrightarrow$ b[j]\\
				\texttt{piv}[$k_j$] $\leftrightarrow$ \texttt{piv}[j]\\
				{\color{gray}\texttt{\# \textbf{Elimination}}}\\
				\For{$k = j+1,\dots,n$}{
					$\ell_{kj} := a_{kj} / a_{jj}$\\
					$a_{kj} = \ell_{kj}$\\
					
					\For{$i = j+1,\ldots, n$}{
						$a_{ki} = a_{ki} - \ell_{kj} a_{ji}$\\
						
					}
				  {\color{gray}\texttt{\#  by hand we also transform b on the fly}}\\
					($b_{k} = b_{k} - \ell_{kj} b_{j}$)
				}
			}
			
		}
			~\\
			{\color{gray}\texttt{\# \textbf{SOLVE}}}\\
			$\ldots$\\
		\caption{In-place Gaussian Elimination with Row Pivoting: Factor and Solve}
		\label{LUwithRowPiv}
	\end{algorithm}	
\end{center}