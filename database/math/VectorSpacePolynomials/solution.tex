{\color{solution}
\begin{enumerate}
	\item 
	\textbf{VR1:} Show, that $(P_n(\mathbb{R}), +)$ is an abelian group.\\
	Let $p(x) =\sum_{k=0}^{n}\alpha_kx^k$, $q(x)=\sum_{k=0}^{n}\beta_kx^k$ and $w(x)=\sum_{k=0}^{n}\gamma_kx^k$ be in $P_n(\mathbb{R})$.
	\begin{itemize}
		\item [(i)] Associativity:$$((p+q)+w)(x)=\sum [(\alpha_k+\beta_k)+\gamma_k]x^k=\sum[\alpha_k+(\beta_k+\gamma_k)]x^k=(p+(q+w))(x)$$
		\item [(ii)] Neutral Element:\\$0:=\sum_{k=0}^{n}0\cdot x^k$, then $\forall~p\in P_n(\mathbb{R})$:$$(0+p)(x)=\sum (0+\alpha_k)x^k=p(x)$$
		\item [(iii)] Inverse element:\\For $p(x)=\sum \alpha_kx^k$ define $-p(x):=\sum (-\alpha_k)x^k$,$$\Rightarrow~~(p+(-p))(x)=0.$$
		\item [(iv)] Commutativity:$$(p+q)(x)=\sum\underbrace{(\alpha_k+\beta_k)}_{\textcolor{blue}{=\beta_k+\alpha_k}}x^k=(q+p)(x)$$
	\end{itemize}
	\textbf{VR2:} Consistency properties: Let $r,s\in\mathbb{R}$.
	\begin{itemize}
		\item [(i)] $((r+s)p)(x)=\sum\underbrace{(r+s)\alpha_k}_{\textcolor{blue}{=r\alpha_k+s\alpha_k}}x^k=(rp)(x)+(sp)(x)$
		\item[] (ii)-(iv)
		$\checkmark$
	\end{itemize}
	\item 
	Let $k<m\in\mathbb{N}$. Then
	\begin{align*}
	&\frac{x^k}{x^m}=x^{k-m}=\frac{1}{x^{m-k}}\stackrel{x\rightarrow +\infty}{\longrightarrow} 0\\
	\Rightarrow~~\forall \alpha_0,\dots,\alpha_m:~~&\frac{\sum_{k=0}^{m-1}\alpha_kx^k}{x^m}=\sum_{k=0}^{m-1}\alpha^k\underbrace{\left(\frac{x^k}{x^m}\right)}_{\textcolor{blue}{\rightarrow 0}}\stackrel{x\rightarrow +\infty}{\longrightarrow} 0.
	\end{align*}
	In particular we can conclude that
	$$
	~~\forall\alpha_0,\dots,\alpha_m:~~\sum_{k=0}^{m-1}\alpha^kx^k\neq x^m~
	$$
	because otherwise limit~$\equiv 1$.
	\item 
	\begin{enumerate}
		\item 
		Linear independence by 2.\\
		Asssume $\exists\alpha_0,\dots,\alpha_n~(m:=\text{max}\{k:\alpha_k\neq 0\})$ not all zero with $\sum_{k=0}^{m}\alpha_kq_k=0$
		\begin{align*}&\Rightarrow~~\sum_{k=0}^{m}\alpha_kq_k=\sum_{k=0}^{m-1}\alpha_kq_k+\alpha_mq_m=0\\
		&\Rightarrow~~\sum_{k=0}^{m}\alpha_kx^k=(-\alpha_m)x^m\\
		&\textcolor{red}{\text{contradiction to 2.}}
		\end{align*}
		\item 
		\begin{align*}
		&\text{span}\{q_0,\dots,q_n\}=P_n(\mathbb{R})~~\text{by definition}\\
		\Rightarrow~~&\text{dim}(P_n(\mathbb{R}))=n+1
		\end{align*}
	\end{enumerate}
	\item 
	Let $p=\sum_{k=0}^{n}\alpha_kq_k$ and $w =\sum_{k=0}^{n}\beta_kq_k$ and $\lambda\in\mathbb{R}$, then:
	\begin{align*}\text{D}(\lambda p+w)(x)&=\left(\sum_{k=0}^{n}(\lambda \alpha_k+\beta_k)x^k\right)'=\sum_{k=0}^{n}(\lambda \alpha_k+\beta_k)kx^{k-1}\\&=\lambda\sum_{k=0}^{n}\alpha_kkx^{k-1}+\sum_{k=0}^{n}\beta_kkx^{k-1}=\lambda \text{D}(p)(x)+\text{D}(q)(x)\end{align*}
	\item[5.] We have:
	\begin{align*}
	D:~P_n(\mathbb{R})\rightarrow P_n(\mathbb{R}),~p=\sum_{k=0}^{n}\alpha_kq_k\mapsto p'&=\sum_{k=0}^{n-1}\alpha_{k+1}(k+1)q_k\\
	\Phi:~P_n(\mathbb{R})\rightarrow\mathbb{R}^{n+1},~p=\sum_{k=0}^{n}\alpha_kq_k\mapsto &(\alpha_0,\dots,\alpha_n)^T\\
	&\textcolor{blue}{=(\pi_0(p),\dots,\pi_n(p))}
	\end{align*}
	$\rightarrow$[$\Phi$ linear since $\pi_j$ are linear (see lecture) and bijective since $\{q_0,\dots,q_n\}$ basis, $\Phi^{-1}:\mathbb{R}^{n+1}\rightarrow P_n(\mathbb{R}),~(\alpha_0,\dots,,\alpha_n)^T\mapsto p=\sum \alpha_kq_k$]\\
	\\
	Now consider: $F:\mathbb{R}^{n+1}\rightarrow\mathbb{R}^{n+1},~F(\alpha):=\Phi\circ D\circ\Phi^{-1}$
	$$
	\begin{matrix}
	\textcolor{red}{(\alpha_o,\dots,\alpha_n)^T}&\mathbb{R}^{n+1}&\xrightarrow{A}&\mathbb{R}^{n+1}&\textcolor{red}{(\alpha_1,2\alpha_2,3\alpha_3,\dots,n\alpha_n,0)=:\beta}\\
	&\textcolor{blue}{\{e_1,\dots,e_{n+1}\}}& &\textcolor{blue}{\{e_1,\dots,e_{n+1}\}}& \\
	&\downarrow \Phi^{-1}& &\Phi\uparrow& \\
	\textcolor{red}{p=\sum_{k=0}^{n}\alpha_kq_k}&P_n(\mathbb{R})&\overrightarrow{D}&P_n(\mathbb{R})&\textcolor{red}{p'=\sum_{k=0}^{n}\beta_kq_k}\\
	&\textcolor{blue}{\{q_1,\dots,q_{n+1}\}}& &\textcolor{blue}{\{q_1,\dots,q_{n+1}\}}& 
	\end{matrix}
	$$
	\begin{align*}
	F(\alpha)&=(\Phi\circ D\circ\Phi^{-1})(\alpha)\\
	&=(\Phi\circ D)\left(\sum_{k=0}^{n}\alpha_kq_k\right)\\
	&=\Phi\left(\sum_{k=0}^{n-1}\alpha_{k+1}(k+1)q_k\right)\\
	&=(\alpha_1,2\alpha_2,3\alpha_3,\dots,n\alpha_n,0)^T
	\end{align*}
	To obtain the matrix representation we have to evaluate $F$ on the standard basis $\{e_1,\dots,e_{n+1}\}$:
	$$
	A=\begin{pmatrix}
	|& &|\\
	F(e_1)&\dots&F(e_{n+1})\\
	|& &|
	\end{pmatrix}
	=\begin{pmatrix}
	0&\textcolor{blue}{1}&0&0&\cdots&0\\
	0&0&\textcolor{blue}{2}&0&\cdots&0\\
	\vdots&\vdots&0&\textcolor{blue}{\ddots}&\ddots&\vdots\\
	0&0&\cdots&\ddots&\textcolor{blue}{n-1}&0\\
	0&0&0&\cdots&0&\textcolor{blue}{n}\\
	0&0&0&\cdots&0&0
	\end{pmatrix}
	$$
\end{enumerate}
}