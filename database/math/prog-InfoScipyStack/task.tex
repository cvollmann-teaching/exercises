{\color{navy}
\textbf{SCIENTIFIC COMPUTING WITH PYTHON}\\~\\
%
Scientific computing in Python is done using the \textbf{SciPy Stack} (https://scipy.org/). This term is used for a collection of packages developed for scientific computing. From this collection we will mainly use the following three packages during this lecture:
  % 
\begin{itemize}
\item \textbf{Numpy} (the basis)\\
    -- provides the data type `numpy.ndarray` (e.g., for matrices and vectors)\\
    -- contains a huge amount of tools to perform all sorts of array manipulation\\
%    
%
\item \textbf{SciPy} (the core)\\
    -- builds upon numpy\\
    -- contains a huge amount of numerical methods (solving linear systems, optimization, integration, interpolation,...)\\
    %
\item \textbf{Matplotlib} (the visualizer)\\
    -- allows to visualize results with high quality grafics (plots in 2d and 3d, images and videos,...)  \\
%
\end{itemize}
%
%
Remarks:
\begin{itemize}
 \item \textbf{Tutorial:} the official one can be found here: https://scipy-lectures.org/. The first chapter in this tutorial contains already enough to do the programming parts in this lecture. 
% 
 \item \textbf{Installation:} typically not needed! They are usually contained in the system-side Python installation or come along with the respective distribution (e.g., anaconda)
%
 \item \textbf{Import convention:} a program often starts with\\~\\
%
\verb|import numpy as np|\\
\verb|import scipy as sp| \\
\verb|import matplotlib as mpl|\\
\verb|import matplotlib.pyplot as plt|\\
\end{itemize}
}
