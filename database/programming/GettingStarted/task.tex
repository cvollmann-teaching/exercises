\begin{enumerate}
	\item Öffnen Sie ein Juypter Notebook und führen Sie den Befehl
	\begin{verbatim}
		print("Hello World")
	\end{verbatim}
	aus.
	\item Fügen Sie mit dem \verb|+|-Button eine weitere Zelle hinzu und führen Sie die Befehle
	\begin{verbatim}
		import sys
		print("Python Version")
		print(sys.version)
	\end{verbatim}
	aus.
	\item Erstellen Sie eine neue Zelle. Ändern Sie die Zelle zum Typ \verb|Raw| und füllen Sie ihn testweise mit einem beliebigen Pseudocode.
	\item  Erstellen Sie eine neue Zelle. Ändern Sie die Zelle zum Typ \verb|Markdown| und versuchen Sie die Formel
	$$
		\sum_{i=1}^n i = n(n+1)/2
	$$
	hervorzubringen.
\end{enumerate}