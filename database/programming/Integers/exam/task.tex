% !TeX spellcheck = de_AT_frami
!!!Separate Aufgaben zu: int und float!!!
\begin{enumerate}
	\item Berechnen Sie die folgenden Summen im entsprechenden Stellenwertsystem.
		\begin{enumerate}
			\item  $(10101010)_2$ + $(11100000)_2$
			\item $(23)_7$ + $(17)_7$
			\item $(19)_{h}$ + $(12)_{h}$ (h für \textit{hexadezimal})
		\end{enumerate}
	!! Ziffernvorrat angeben !! ((b) ist ungültig)
	\item 
	\begin{enumerate}
		\item Wandeln Sie die folgenden Zweierkomplementzahlen ins Dezimalsystem um.
		\begin{enumerate}
			\item $(10101010)_{\mathbb{Z}}$
			\item $(11100000)_{\mathbb{Z}}$
		\end{enumerate}
		\item Hierbei handelt es sich um die Bitfolgen von oben. Angenommen Sie betrachten nur die ersten 8 Bits des Ergebnisses der Summe (d.h. das höchstwertigste Bit wird ggf. abgeschnitten). Bestimmen Sie das letzte Übertragsbit und das Überlaufsbit.  Treffen Sie damit Aussagen über die Gültigkeit des abgeschnittenen Ergebnisses: Interpretieren Sie dabei einmal alle Bitfolgen als Zweierkomplementzahlen und einmal alle als vorzeichenlose Binärzahlen.
	\end{enumerate}
	\item Stellen Sie die Zahlen $x = 7$ und $z = -7$ als Zweierkomplementzahl der Länge $N = 4$ sowie $N = 6$ dar.
\end{enumerate}
