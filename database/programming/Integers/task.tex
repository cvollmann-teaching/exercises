\textbf{Stellenwertsysteme zur Basis $2^n$ und Zweierkomplementzahlen}
\begin{enumerate}
	\item Berechnen Sie die folgenden Summen im entsprechenden Stellenwertsystem.
		\begin{enumerate}
			\item $(1010 0110 0111)_2$ + $(0111 0100 0101)_2$
			\item $(5147)_8$ + $(3505)_8$
			\item $(A47)_h$ + $(745)_h$ 
		\end{enumerate}
	\item Wandeln Sie die folgenden 2er-Komplementzahlen ins Dezimalsystem um.
		\begin{enumerate}
			\item $(01011)_{\mathbb{Z}}$
			\item $(10011)_{\mathbb{Z}}$
		\end{enumerate}
	\item Stellen Sie die Zahlen $x = 5$ und $z = -5$ als 2er-Komplementzahl der Länge $N = 4$ sowie $N = 6$ dar.
	\item Gegeben seien die Bitfolgen der Länge 8: $a = (11100011)$ und $b = (10110001)$. Addieren Sie die beiden Bitfolgen einmal interpretiert als Binärzahl und einmal als 2er-Komplementzahl. Sind die Ergebnisse gültig oder entsteht ein Überlauf in dem jeweiligen Format? Begründen Sie kurz Ihre Antwort.
	\item Ordnen Sie alle 2er-Komplementzahlen der Länge $N = 3$ gemäß ihres Rests modulo $2^N$. Was beobachten Sie?
	\item Gegeben ist die Binärzahl $(a_{N-1}\dots a_1 a_0)$, $N \geq 1$, $a \in \lbrace 0,1 \rbrace$. Welches Bit gibt Auskunft darüber, ob die dargestellte Zahl gerade bzw. ungerade ist? Begründen Sie kurz Ihre Antwort.
\end{enumerate}