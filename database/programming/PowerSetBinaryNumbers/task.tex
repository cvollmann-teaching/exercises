\textbf{Binärzahlen und die Potenzmenge}
\begin{enumerate}
	\item Schreiben Sie eine Python-Funktion \texttt{toBinary(k)}, welche die Binär-Ziffern einer natürlichen Zahl $k$ in Form einer Liste zurückgibt.
	\item Schreiben Sie eine Python-Funktion \texttt{powerSet(inputList)}, welche die Potenzmenge einer Liste \verb|inputList| als Liste von Listen zurückgibt.
\end{enumerate}
\textbf{Beispiel}\\
\texttt{powerSet([1,2,3])}\\
\verb|>>> [[], [1], [2], [2, 1], [3], [3, 1], [3, 2], [3, 2, 1]]|\\
~\\
\textit{Hinweis:} Die Potenzmenge von \verb|inputList| hat $2^m$ Elemente für $m=$\verb|len(inputList)|. Versuchen Sie einen Weg zu finden die Auswahl der Teilmengen über Binär-Ziffern zu steuern.