\textbf{Half Precision: \texttt{s10e5}}\\
Betrachten Sie die binären Gleitkommazahlen gemäß der Parameter \texttt{s10e5}.
\begin{enumerate}
	\item Ermitteln Sie den Verschiebewert $B$ (\texttt{BIAS}) für den Exponenten gemäß des IEEE-754 Standards, sowie den Exponentenwertebereich $e_{\text{min}}$ und $e_{\text{max}}$.
	\item Mit vielen Bits wird eine Gleitkommazahl hier gespeichert? Erklären Sie die Bedeutung der einzelnen Bits und wie daraus die Gleitkommazahl gebildet wird.
	\item Bestimmen Sie die relative Maschinengenauigkeit \texttt{macheps}.
	\item Geben Sie die Bitmuster mit zugehörigem (ungefähren) Zahlenwert an:
	\begin{enumerate}
		\item Signed zero und signed infinity:  $\pm 0$, $\pm \infty$
		\item Kleinste und größte positive normale Zahl
		\item Kleinste und größte positive denormalisierte Zahl (\textit{subnormals})
		\item Kleinste und größte positive Nichtzahl (\texttt{NaN})
		\item \texttt{0011110000000001}
		\item $\texttt{fl}(-4)$
		\item $\texttt{fl}(\frac{2}{3})$, wobei $\frac{2}{3} = (0.\overline{10})_2=(0.10101010\ldots)_2$
	\end{enumerate}
\end{enumerate}