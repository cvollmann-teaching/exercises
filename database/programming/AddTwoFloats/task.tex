\textbf{Sicherstellung des richtigen Formats (\texttt{float()})}
Wir bezeichnen mit \verb|add(a, b)| eine Python Funktion, die die Summe zweier Zahlen $a$ und $b$ berechnet.
\begin{enumerate}
	\item Schreiben Sie eine Funktion \verb|test()|, die die Funktion \verb|add(a, b)| für die Eingabepaare \verb|a="2.0", b="3.1"| und \verb|a=3, b=3| testet und einen Fehler produziert, falls das Ergebnis nicht korrekt ist.
	\item Implementieren Sie die Funktion \verb|add(a, b)|. Dabei sollen die Eingabewerte zunächst
	mit der Funktion \verb|float()|  umgewandelt werden.
	\item Testen Sie als Input der Funktion auch \verb|a="2,3", b="b"|. Was ist der Vorteil der Nutzung von \verb|float()| neben der Umwandlung?
\end{enumerate}

\textbf{Hintergrund: } Ihnen steht wahrscheinlich die Implementierung komplexer Algorithmen oder Programme bevor. Wenn die einzelnen Teile ihres Programms schlau überprüfen, ob sie sinvolle Eingaben erhalten haben, werden Sie sich selbst bei einer Fehlersuche sehr dankbar sein. \\