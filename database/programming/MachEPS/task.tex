\textbf{Mantissenlänge und relative Maschinengenauigkeit \texttt{macheps}}
\begin{enumerate}
	\item Gegeben seien $z = \pm m \cdot2^e,~ e \in \lbrace e_{\text{min}}, \dots, e_{\text{max}}\rbrace$ mit $m = (1\textbf{. }a_{-1}~a_{-2}~a_{-3} \dots)_2$.
	Zudem sei $\varepsilon > 0$ gegeben. Angenommen Sie wollen $z$ als binäre Gleitkommazahl $\overline{z}=fl(z)$ mit Mantissenlänge $N$ darstellen, wobei $fl$ gemäß \textit{roundest-to-nearest} rundet. Wie groß müssen Sie $N$ wählen, sodass für den relativen Fehler 
	\begin{align*}
	\frac{|z - \overline{z}|}{|z|} < \varepsilon
	\end{align*}
	gilt? Berechnen Sie $N$ beispielhaft für $10^{-7}$ und $10^{-16}$.
	
	\item Schreiben Sie ein Python-Programm, dass die Maschinengenauigkeit ermittelt, indem es einen Wert $\varepsilon$ so lange halbiert, bis
	$1 + \varepsilon > \varepsilon$ als Falsch gewertet wird. Geben Sie diese Zahl an.
\end{enumerate}