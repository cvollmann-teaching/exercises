\textbf{Heron's Algorithmus/ Babylonisches Wurzelziehen}\\
Berechnen Sie für eine beliebige nichtnegative Zahl $a\ge 0$ die Wurzel $\sqrt{a}$ bis auf einen (relativen) Fehler von $10^{-10}$ nach der folgenden iterativen Methode:
$$ x_{n+1}=\frac{1}{2} \left(\frac{x_n^2+a}{x_n}\right) \nonumber.$$ 
\begin{enumerate}
	\item Schreiben Sie dafür eine Python Funktion \texttt{heron(a, x0, tol)}.
	\item Wählen Sie verschiedene Anfangswerte $x_0$ und beobachten Sie das Konvergenzverhalten, indem Sie zum Beispiel die Fehler $|x_n-\sqrt{a}|$ in einer Liste abspeichern.
	\item Erweitern Sie die Funktionsschnittstelle um einen Parameter \texttt{maxiter} (\texttt{heron(a, x0, tol, maxiter)}). Brechen Sie (zusätzlich zum Fehlertoleranzkriterium) die Iteration ab, sobald die Anzahl der Schleifendurchläufe den Wert \texttt{maxiter} erreicht hat.
\end{enumerate}

\textit{Hinweis:} Benutzen Sie die built-in Funktion \texttt{abs()} und den Python-Wert \texttt{a ** 0.5} zur Fehlermessung. 
