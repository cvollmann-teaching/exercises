{
\color{solution}
Since $A$ is symmetric we can exploit the estimate $$\|x_k-x\|_2 \leq \rho(M)^k\|x_0-x\|_2,$$ where $\rho(M)$ denotes the spectral radius of the iteration matrix.
Thus we want to find $m \in \mathbb{N}$ such that for all $k \geq m$, we have
$$\rho(M)^k \leq \varepsilon. $$
Applying logarithm we find that 
$$m =\frac{ \text{log}(\varepsilon)}{ \text{log}(\rho(M))}. $$
Therefore let us compute the spectral radius for this case and each method. Consider the splitting $A = L + D + U$.\\~\\
\textbf{Richardson:}\\
Here we have
\begin{align*}
	M_R = I - A = \begin{pmatrix}
		-2&1\\1&-2
	\end{pmatrix}.
\end{align*}
Thus 
$$0 = \text{det}( M_R-\lambda I) = (2+\lambda)^2 - 1 $$
gives $\lambda \in \{-1, -3\}$, so that $\rho(M_R) = 3$. Therefore Richardson without relaxation does not converge!\\~\\
\textbf{Jacobi:}\\
Here we have
\begin{align*}
	M_J = I - D^{-1}A 
	= I-\begin{pmatrix}
		\tfrac{1}{3}&0\\0&\tfrac{1}{3}
	\end{pmatrix}
	\begin{pmatrix}
		3&-1\\-1&3
	\end{pmatrix}
=
\begin{pmatrix}
	0&\tfrac{1}{3}\\\tfrac{1}{3}&0
\end{pmatrix}.
\end{align*}
Thus 
$$0 = \text{det}( M_J-\lambda I) = \lambda^2 - \tfrac{1}{9}$$
gives $\lambda \in \{-\tfrac{1}{3}, \tfrac{1}{3}\}$, so that $\rho(M_J) = \tfrac{1}{3}$. Therefore Jacobi without relaxation does converge and 
$$m =\frac{ \text{log}(\varepsilon)}{ \text{log}(\tfrac{1}{3})}\approx 13. $$
\textbf{Gauß-Seidel:}\\
Here we have
\begin{align*}
	M_G = I - (D+L)^{-1}A
\end{align*}
where 
$$(D+L)^{-1} = \begin{pmatrix}
\tfrac{1}{3}&0\\\tfrac{1}{9}&\tfrac{1}{3}
\end{pmatrix}  $$
so that
\begin{align*}
	M_G 
	= I-\begin{pmatrix}
		\tfrac{1}{3}&0\\\tfrac{1}{9}&\tfrac{1}{3}
	\end{pmatrix}
	\begin{pmatrix}
		3&-1\\-1&3
	\end{pmatrix}
	=
	\begin{pmatrix}
		0&\tfrac{1}{3}\\0&\tfrac{1}{9}
	\end{pmatrix}.
\end{align*}
Thus 
$$0 = \text{det}(M_G -\lambda I) = -\lambda (\tfrac{1}{9}-\lambda)$$
gives $\lambda \in \{0, \tfrac{1}{9}\}$, so that $\rho(M_G) = \tfrac{1}{9}$. Therefore Gauß-Seidel without relaxation does converge and 
$$m =\frac{ \text{log}(\varepsilon)}{ \text{log}(\tfrac{1}{9})}\approx 7. $$
}