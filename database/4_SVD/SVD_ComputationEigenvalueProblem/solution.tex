{\color{solution}
\begin{enumerate}
    \item Because $S$ is symmetric.
    \item 
We already identify $r$ eigenpairs for $S$, namely, 
$$(\sigma_1,\begin{pmatrix}
u_1\\ v_1
\end{pmatrix}), \ldots, (\sigma_r,\begin{pmatrix}
u_r\\ v_r
\end{pmatrix}),$$ where $(\sigma_i,\begin{pmatrix}
u_i\\ v_i
\end{pmatrix})$ are the $r$ singular values and vectors of $A$, respectively.
~\\~\\
Also, we easily find that 
$$(-\sigma_1,\begin{pmatrix}
-u_1\\ v_2
\end{pmatrix}), \ldots, (-\sigma_r,\begin{pmatrix}
-u_r\\ v_r
\end{pmatrix})$$
 are eigenpairs of $S$.\\ 
 For the remaining $(m-r)+(n-r)$ eigenpairs take orthonomal bases $u_{r+1},\ldots, u_m\in\text{ker}A^\top$ and $v_{r+1},\ldots, v_n\in\text{ker}A$, then the $(0,\begin{pmatrix}
	u_i\\ 0
\end{pmatrix})$ and $(0,\begin{pmatrix}
0\\ v_i
\end{pmatrix})$ give the remaining eigenpairs (with eigenvalue $0$).
\item Take the all eigenpairs $(\sigma, w)$ of $S$ corresponding to positive eigenvalues $\sigma >0 $. Split the eigenvectors $w =(u,v)$.
\end{enumerate}
}
