% !TeX spellcheck = en_US
\textbf{Vector Space of Polynomials}\\
%{\color{navy}
%Let $\F$ be a field. A set $V$ together with a mapping $+$ 
%(sum) and a mapping $\cdot$ (scalar multiplication) with 
%\begin{eqnarray*} \begin{array}{r@{~}lr@{~}l} 
%		+: V \times V & \to V & \cdot : \F \times V & \to V \\ 
%		(v,w)       & \mapsto v+ w    & (\lambda, v) & \mapsto \lambda \cdot v  
%\end{array} \end{eqnarray*} 
%is called \emph{\textbf{vector space (or linear space)} over $\mathbb{F}$}, if the following axioms \textbf{VR1} and \textbf{VR2} hold: 
%
%\begin{itemize} 
%	\item[\textbf{VR1}] $(V,+)$ is a commutative (or abelian) group with neutral element $0$, i.e., 
%	\begin{itemize}
%		\item[\textbf{G1}]Associativity: $\forall v_1, v_2,v_3\in V:~ v_1+(v_2+v_3)=(v_1+v_2)+v_3$
%		\item[\textbf{G2}] Neutral element: $\forall v \in V: ~v + 0 = v$
%		\item[\textbf{G3}] Inverse element: $\forall v \in V ~\exists_1 (-v) \in V:~ v+(-v)=0$
%		\item[\textbf{G4}] Commutativity: $\forall v_1, v_2 \in V:~ v_1 + v_2 = v_2+v_1$
%	\end{itemize}
%	\item[\textbf{VR2}] The scalar multiplication is consistent/compatible with 
%	$(V,+)$ in the following way:  
%	
%	for $\lambda,\mu \in \F$, $v,w \in V$ it holds that  
%	\begin{itemize} 
%		\item[(i)] $(\lambda + \mu)\cdot v = \lambda \cdot v + \mu \cdot v$ 
%		\item[(ii)] $\lambda \cdot (v+w) = \lambda \cdot v + \lambda \cdot w$ 
%		\item[(iii)] $\lambda \cdot (\mu \cdot v) = (\lambda \cdot \mu) \cdot v$  
%		\item[(iv)] $1 \cdot v = v$ 
%	\end{itemize} 
%\end{itemize}
%Furthermore, let $v_1,\ldots, v_n \in V$, then with the summation and scalar multiplication we can more generally define the \textbf{span} as
%$$\text{span}(v_1,\ldots, v_n)= \{\sum_{i=1}^n \lambda_i v_i: \lambda_i \in \mathbb{F}\}.$$
%Further we say that $v_1,\ldots, v_n \in V$ are \textbf{linearly independent} if 
%$$\sum_{i=1}^n \lambda_i v_i = 0 ~~~\Rightarrow ~~~\lambda_i = 0 ~~\forall~i. $$
%If $v_1,\ldots, v_n \in V$ are linearly independent and $\text{span}(v_1,\ldots, v_n) = V$, then we call $v_1,\ldots, v_n$ a \textbf{basis of $V$}.\\
%A mapping $f\colon V_1 \to V_2$ between two vector spaces is called \textbf{linear}, if $$f(\lambda \cdot_1 v +_1 w) = \lambda \cdot_2 f(v) +_2 f(w)$$
%for all $v,w\in V$. Here $+_1,\cdot_1$ and $+_2,\cdot_2$ denote the summation and scalar multiplication defined on $V_1$ and $V_2$, respectively. Examples are $\mathbb{R}^n$ and $\mathbb{R}^{m\times n}$ with the usual vector/matrix sum ``$+$'' and scalar multiplication ``$\cdot$''.
%}
%~\\
%Now we consider another example of a vector space: 
Let $n \in \mathbb{N}$ and $P_n(\mathbb{R})$ be the set of all polynomials of degree $\leq n$ on $\mathbb{R}$, i.e., the set $P_n(\mathbb{R})$ contains all functions $p:\mathbb{R} \rightarrow \mathbb{R}$ of the form
\begin{align*}
p(x) = \sum_{k=0}^n \alpha_k x^k
\end{align*}
for some $\alpha_0,\alpha_1, \dots,\alpha_n \in \mathbb{R}$. We define a summation and scalar multiplication:
\begin{align*}
+&\colon P_n(\mathbb{R}) \times P_n(\mathbb{R}) \to P_n(\mathbb{R}),~(p+q)(x) := p(x) + q(x),\\
\cdot &\colon  \mathbb{R} \times P_n(\mathbb{R}) \to P_n(\mathbb{R}),\quad\quad ~~(r\cdot p)(x) := r\cdot p(x) .
\end{align*}
\begin{enumerate}
	\item \textbf{VR axioms:} Please show that $ P_n(\mathbb{R})$ together with the above defined summation and scalar multiplication forms a vector space.
%	\textit{Hint:} Check \textbf{VR1} and \textbf{VR2} with $V = P_n(\mathbb{R})$.
	\item Let $k<m \in \mathbb{N}$. Compute 
	$$
	\lim\limits_{x\rightarrow\infty} \frac{x^k}{x^m}, ~~\text{ and }~~ \lim\limits_{x\rightarrow\infty}  \frac{\sum_{k=0}^{m-1}\alpha_k x^k}{x^m}
	$$
	for arbitrary $\alpha_0, \dots, \alpha_m \in \mathbb{R}$.
	\item \textbf{Monomials form a basis:} Please show that the set $B := \lbrace q_0, \dots, q_n \rbrace$ with
	\begin{align*}
	q_k: \mathbb{R} \rightarrow \mathbb{R}, \hspace{.2cm} x \mapsto x^k,
	\end{align*} 
	is a basis of $P_n(\mathbb{R})$. What is the dimension of the vector space $P_n(\mathbb{R})$?
	\textit{Hint:  Part 2. basically provides the proof of linear independence.}
	\item \textbf{Derivative as linear operator:} Show that the operator $\mathcal{D}:P_n(\mathbb{R}) \rightarrow P_{n-1}(\mathbb{R}), \hspace{.2cm} p \mapsto p'$, 
	which maps a polynomial to its first derivative, is an $\mathbb{R}$-linear function.\\
	\textit{Hint:  You can use that for $p(x) = \sum_{k=0}^n \alpha_k x^k$ we have $Dp(x)=p'(x) = \sum_{k=0}^{n-1} \alpha_{k+1} (k+1)x^k$. }					
	
	\item[5.] \textbf{Matrix representation of the derivative:} Derive the matrix representation of $\mathcal{D}$ with respect to the bases of monomials, i.e., derive the matrix $\mathcal{M}^{\{q_0,\ldots,q_n\}}_{\{q_0,\ldots,q_{n-1}\}}(\mathcal{D})$.
	
%Let $\Phi$ be the linear, invertible function which maps a polynomial to its coefficients (coordinates in the above basis), i.e.,
%	$$\Phi: P_n(\mathbb{R}) \rightarrow \mathbb{R}^{n+1}, \hspace{.2cm} \sum_{k=0}^n \alpha_k x^k \mapsto (\alpha_0, \dots, \alpha_n).$$ 
%	Please remark shortly why $\Phi$ is bijective.
%	What is the matrix representation of the linear function
%	$ F:\mathbb{R}^{n+1} \rightarrow \mathbb{R}^{n+1}$ defined by $$F :=\Phi \circ \mathcal{D} \circ \Phi^{-1}  $$ with respect to the standard basis $\lbrace e_1, \dots, e_{n+1}\rbrace$? More precisely, derive the matrix $A \in\mathbb{R}^{(n+1)\times (n+1)}$ defined by
%	$$
%	A:=\begin{pmatrix}
%	|& &|\\
%	F(e_1)&\dots&F(e_{n+1})\\
%	|& &|
%	\end{pmatrix}.
%	$$
	
%	\textit{Note: } This is a bonus exercise. %The representation refers to the one in Thm 9.8 and the successive remark (as $\Pn$ is a general vector space).
	
\end{enumerate}