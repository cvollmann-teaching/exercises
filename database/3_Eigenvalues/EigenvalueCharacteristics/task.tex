\textbf{Eigenvalue Characteristics}\\
We denote the identity matrix of $\mathbb{F}^{n \times n}$ by $I$. Let $A \in \mathbb{F}^{n\times n}$ be a matrix. Some value $\lambda \in \mathbb{C}$ 
is called eigenvalue of $A$, if there is a vector $v \neq {0}$ in $\mathbb{F}^{n}$ such that
\begin{align*}
(A - \lambda I)v = {0},
\end{align*}  
where ${0} \in \mathbb{F}^{n}$ denotes the zero vector. Use the above definition to prove the following assertions.
\begin{enumerate}
	\item If $A$ is invertible, then for all eigenvalues $\lambda$ of $A$ we have $\lambda \neq 0$ and $\frac{1}{\lambda}$ is an eigenvalue of $A^{-1}$.
	\item If $\lambda$ is an eigenvalue of $A$, then $(\lambda - \alpha)$ is an eigenvalue of $(A - \alpha I)$ for any $\alpha \in \mathbb{C}$.
	\item If $\lambda$ is an eigenvalue of $A$, then $\lambda$ is also an eigenvalue of $Q^\top A Q$ for any orthogonal matrix $Q \in \mathbb{F}^{n \times n}$.
\end{enumerate}